\section{The stabilizer encoding}
\label{sec:stabilizer}



\subsection{Analyzing $V_{UPC}$ with general strategies}

We now carry out the second stage of the analysis, where we analyze the value of the protocol where the VTM is $V_{UPC}$ instead of $V_H$. 

\paragraph{Provers} The set $\cP_r$ of provers that $V_{UPC}$ interacts with has size $\kappa_r = \kappa_{r-1} + 7$. What was originally prover $PV$ is now represented by provers $PV_1,\ldots,PV_7$. 

\subsection{Stabilizer check}

\begin{center}
\begin{mdframed}
    Input: $(1^n,w,G,V,M,r)$ \\
	\begin{enumerate}
		\item Do the multi qubit stabilizer test.
	\end{enumerate}    
\end{mdframed}
\end{center}
\begin{figure}[H]
\caption{Stabilizer Check}
\label{fig:stabilizer_check}
\end{figure}

\paragraph{The high level} The Stabilizer check is performed with the provers $PV_1,\ldots,PV_7$, and checks that they share a $7$-qubit stabilizer encoding of the registers $\what{\sO}, \what{\sK}_1,\what{\sK}_2, \what{\sC}, \what{\sV}, \what{\sM}, \what{\sE}_1, \what{\sE}_2$. 

% the prover $PV$ (which is supposed to play the role of the verifier of $UPC_{N,r-1}$) has the $\what{\sV}_{in}$ register initialized to the input $(G,V,M)$, and $\what{\sO} \what{\sV}_{work} \what{\sM} \what{\sK}_1 \what{\sK}_2$ set to zeroes. Furthermore, the subroutine will check that the $PV$ shares the maximally entangled state with $PP_i$ in the $\what{\sE}_{1,i} \what{\sE}_{2,i}$ registers.

%the $PP_i$ prover has the message register $\sM_{r-1,i}$ set to all zeroes.

\paragraph{Question types} 

\paragraph{Honest Input Check strategy}
A strategy $\strat = (\rho,\{M_i\})$ with shared state $\ket{\psi}_{\sC \sP \sR}$ is an \textbf{honest Stabilizer strategy} if the measurement operators of the provers $PV_1,\ldots,PV_7$ are all honest Pauli measurements, the registers $\sP_{V,1},\ldots,\sP_{V,7}$ contain a distributed encoding of the registers $\what{\sO}, \what{\sK}_1,\what{\sK}_2, \what{\sC}, \what{\sV}, \what{\sM}, \what{\sE}_1, \what{\sE}_2$.

\hnote{Probably better ways of saying all these things}


\begin{lemma}	
\label{lem:stabilizer_check}
\leavevmode
\begin{enumerate}
	\item (\textbf{Completeness}) An honest Stabilizer Check strategy passes the $\textbf{Stabilizer Check}$ subprotocol with probability $1$. 
	\item (\textbf{Soundness}) There exists a $\delta = \poly[N;\eps]$ such that all strategies $\strat$ that pass the $\textbf{Stabilizer Check}$ subprotocol with probability at least $1 - \eps$ are $\delta$-isometric to an honest Stabilizer strategy.
\end{enumerate}
\end{lemma}


\subsection{The other subroutines}
The \textbf{Gate check}, \textbf{Input check}, and \textbf{Output check} subroutines are exactly the same as \textbf{Gate2 check}, \textbf{Input2 check}, and \textbf{Output2 check} except whatever question was supposed to be sent to $PV$ is now sent simultaneously to $PV_1,\ldots,PV_r$. Answers that were meant to be from $PV$ are now computed by taking the XOR of all the answers from $PV_1,\ldots,PV_7$. 


\subsection{Prover simulation}

\paragraph{Honest Single-Prover Pauli strategy} For $r\geq 0$ we say that an $(r+1)$-prover strategy $(\ket{\psi},\{M_i\})$ for an extended nonlocal game is an ($S$-qubit) \emph{honest single-prover Pauli strategy} (implicitly, for the $t$-th prover, for some $t\in\{1,\ldots,r+1\}$) if the register of $\ket{\psi}$ that is associated with the $t$-th prover consists of $S$ qubits, and moreover on any question the answer bits returned by the $t$-th prover are obtained by performing a set of commuting observables (one for each answer bit), each of which can be expressed either as a tensor product of at most two $\sigma_X$ Pauli observables, or a tensor product of at most two single-qubit $\sigma_Z$ Pauli observables.



\begin{definition}
Call an extended nonlocal game an ($S$-qubit) \emph{Single Pauli Prover game} if the following holds. The game has $(r+1)$ provers. The first prover is called the ``Pauli prover''. In the game, queries take the form $Q=(q_P,q_1,\ldots,q_r)$, where the question $q_P$ to the Pauli prover is always a triple of the form $(W_1,W_2,W_3)$ where $W_1,W_2,W_3$ are labels for commuting  observables, each of which is a tensor product either of at most two single-qubit $\sigma_X$ Pauli observables, or at most two single-qubit $\sigma_Z$ Pauli observables. 
\end{definition}


Let $G$ be an $(r+1)$-prover Single Pauli Prover game such that $r\geq 7$. We define an $r$-prover \emph{Simulated Pauli Prover game} $G'$ by using $7$ out of the $r$ provers to simulate the Pauli prover in $G$ using a technique similar to the ``code-check'' test in~\cite{NV17} (see also~\cite{Ji}). We then show that any $r$-prover strategy in $G'$ can be ``decomposed'' into an $(r+1)$-prover strategy in $G$ with similar success probability. 

To achieve this we first define the kind of codes we need and introduce an appropriate multi-qubit test for constant-weight Pauli observables. 


\subsubsection{Background material}
\label{sec:codes}

\paragraph{Stabilizer codes}
We consider weakly self-dual  \emph{Calderbank-Shor-Steane (CSS)
  codes}~\cite{CalderbankShor96,Steane96}. Let $C$ be a classical $[k,k']$ linear error-correcting code over $\Fp_2$: $C$ is specified by a generator matrix $H \in \Fp_2^{k\times k'}$ and a parity check matrix $K\in \Fp_2^{(k-k')\times k}$ such that $C = \text{Im}(H)=\ker(K)$. We say that $C$ is weakly self-dual if the dual code $C^\perp$, with generator matrix $K^T$, is such that $C\subseteq C^\perp$; equivalently, $H^T H=0$. To any such  code $C$ we associate a subspace $\mathcal{C}$ of 
$(\C^2)^{\otimes k}$ that is the simultaneous $+1$ eigenspace of a set of stabilizers 
$\{S_{W,j}\}_{W\in\{X,Z\},j\in\{1,\ldots,k'\}}$ such that $S_{W,j}$ is
a tensor product of Pauli $\sigma_W$ observables over $\Fp_2$ in the locations indicated by the
$j$-th column of the generator matrix $H$, i.e.
\[ S_{W,j} = \sigma_W(H_{1j}) \otimes \sigma_W(H_{2j}) \otimes \cdots
  \otimes  \sigma_W(H_{kj}), \]
where $H_{ij}$ is the $(i,j)$-th entry of $G$.
The condition that $H^TH=0$ implies that all the $S_{W,j}$ commute, so that $\mathcal{C}$ is well-defined. 

\begin{example}[Quadratic residue code]
\label{ex:quad_res_code}
The $7$-qudit code is a weakly self-dual CSS code that has $k=7$, $k'=3$, and encodes one qudit over $\Fp_2$.
\end{example}

\paragraph{EPR test} 
By replacing the use of the CHSH test with the Magic Square test in the EPR test from~\cite{chao2016test} we obtain the following test and guarantees. 

\vspace{10pt}
\begin{center}
\begin{mdframed}
    Input: an integer $S$ and a set of $2S$ labels, $\{W_i|\, W\in\{X,Z\},\,i\in\{1,\ldots,S\}\}$.\\
		The verifier performs each of the following with equal probability:
		\begin{enumerate}
		\item Select $i\neq j\in\{1,\ldots,S\}$ and $W,W'\in\{X,Z\}$ uniformly at random. Send $(W_i,W'_j)$ to player A. Select two two-qubit Pauli observables $W^{(2)},W^{(3)}$ that mutually commute and commute with $WW''$. Send the (lexicographically ordered) triple $(W_iW'_j,W^{(2)},W^{(3)})$ to $B$. Receive bits $(a,a')$ from player A and $(b_1,b_2,b_3)$ from player $B$. Accept if and only if $aa'=b_1$.
		\item Select pairwise distinct $i,j,k,\ell\in\{1,\ldots,S\}$, a pair of questions $(q,q') \in \QMS\times\QMS$ in the Magic Square test, and a uniformly random $q''\in\QMS$. Send $(q,(i,j))$ to player A and the unordered pair $(q',(i,j))$ and $(q'',(k,\ell))$ to player B.\footnote{In all cases, if $q$ is of the form $(W,W')$, we label the question $(q,(i,j))$ as $(W_i,W'_j)$, to make it indistinguishable from a question in part 1. of the test.} Accept if and only if the players' answers associated with the query $(q,q')$ would be accepted in the Magic Square test. 
   \end{enumerate}    
\end{mdframed}
\end{center}
\begin{figure}[H]
\caption{$k$-qubit EPR test~\cite{chao2016test}}
\label{fig:epr_test}
\end{figure}

\paragraph{Honest EPR strategy} An \emph{honest $k$-qubit EPR strategy} $\mS = (\ket{\psi}, \{W_i\})$ is a two-player strategy that satisfies the following conditions. In the strategy the players share a state $\ket{\psi} = \ket{\Phi}_{\reg{A}'\reg{B}'} \otimes \ket{\psi}_{\reg{ABR}}$, where player A has registers $\reg{A}$ and $\reg{A}'$, player B has registers $\reg{B}$ and $\reg{B}'$, and register $\reg{R}$ is an arbitrary purifying register. The state $\ket{\Phi}_{\reg{A}'\reg{B}'}$ is an $S$-qubit maximally entangled state. When sent a question of the form $(W_i,W'_j)$, the player determines its two answer bits by measuring $\sigma_W$ on the $i$-th qubit and $\sigma_{W'}$ on the $j$-th qubit. The same holds if there are four $(W_i,W'_j,W''_k,W'''_\ell)$. If the question has the form $(W^{(1)},W^{(2)},W^{(3)})$ for three mutually commuting (at most) two-qubit Pauli observables, the player returns the three bits obtained by simultaneously  measuring the three observables. 

The following follows from the results in~\cite{chao2016test}. 

\begin{theorem}\label{thm:epr-test}
Let $S \geq 4$ be an integer.\footnote{We need $S\geq 4$ to be able to ask the ``confuse'' questions in part 2. of the test.} The $S$-qubit EPR test has the following guarantees. 
\begin{itemize}
\item (Complexity) Questions in the test are $O(\log S)$-bit long. Answers are $O(1)$-bit long. 
\item (Completeness) Any honest $S$-qubit EPR strategy succeeds with probability $1$ in the test. 
\item (Soundness) For any $\eps\geq 0$ there is a $\delta = \poly[S;\eps]$ such that any strategy that succeeds with probability at least $1-\eps$ in the test is $\delta$-isometric to a honest $S$-qubit EPR strategy. 
\end{itemize}
\end{theorem}

%\begin{theorem}[Pauli braiding test]\label{thm:pbt}
  %For any integer $S\geq 2$ there exists a two-player nonlocal game with $O(S)$-bit questions and $O(1)$-bit answers such
  %that any strategy
  %that has success probability at least $1-\eps$ for some
  %$\eps\geq 0$ must be based on observables $A(x),A(z),A(x,z)$ and an
  %entangled state $\ket{\psi}_{AB}$ such that up to local isometries
  %$$A(x) \approx_{\delta} \otimes_i \sigma_X^{x_i}, \qquad A(z) \approx_{\delta} \otimes_i \sigma_Z^{z_i},\qquad\text{and}
%\qquad \ket{\psi}_{AB} \approx_{\delta} \ket{\Phi}_{AB}^{\otimes n},$$
%where $\delta=\poly(\eps)$. 
%\end{theorem}

\subsubsection{Simulated Pauli Prover game}

For convenience, fix a self-dual CSS code with $k=7$, such as the code from Example~\ref{ex:quad_res_code}. Let $H$ be the generator matrix of the code. Let $G$ be an $S$-qubit $(r+1)$-prover Single Pauli Prover game such that $r\geq 7$. The associated Simulated Pauli prover game is an $r$-prover extended nonlocal game that is described in Figure~\ref{fig:simpauli_check}. Before the game starts, $7$ out of the $r$ provers are designated to be the ``simulated Pauli prover''. Once the game has started, the verifier splits the provers that constitute the simulated Pauli prover into two groups. One prover, indexed by $t\in\{1,\ldots, r\}$, is chosen
at random and called the ``special prover''. The remaining $6$ provers, indexed by $i_1,\ldots,i_6 \in \{1,\ldots,r\}\backslash\{t\}$, are
jointly called the ``composite prover''. In general a
prover is not told whether it is the special prover, or a composite prover. The description of the game involves notions of ``composite query'' and ``composite answer'' that are defined as follows.  

\begin{definition}[Composite queries and answers]\label{def:queries}
Let $W$ be an $S$-qubit Pauli observable.
\begin{enumerate}
\item The composite query associated with $W$, denoted $\comp{W}$, is
  obtained by sending each prover forming the composite prover the question $W$.
\item 
Given answers $(A_{i_j})_{i\in\{1,\ldots,6\}}$ from the $6$ provers
  forming the composite prover, the composite answer $\comp{A}$ is
  obtained by selecting a uniformly random vector $v$ in the column
  span of $H$ such that $v_7=1$, and computing the sum $\comp{A} =
  \sum_{j\in\{1,\ldots,6\}} v_j A_{i_j}$.
\end{enumerate}
\end{definition} 


\begin{center}
\begin{mdframed}
    Input: an $S$-qubit $(r+1)$-prover Single Pauli Prover game $G$ \\
		There are $r$ provers, $7$ of which have been designated as the ``simulated Pauli prover''. Among those $7$ provers, the verifier chooses one at random to be the ``special prover'' $P_t$, for $t\in\{1,\ldots,r\}$. The remaining $6$ provers, $P_{i_1},\ldots,P_{i_6}$, are jointly referred to as the ``composite prover''. The verifier performs either of the following with equal probability: 
	\begin{enumerate}
		\item \emph{(Stabilizer test)} The verifier generates a query $(q,q')$ in the $S$-qubit EPR test.  The verifier sends $q$ to the special prover, and distributes $\comp{q'}$ to the composite prover ($q'$ is in general a pair or triple of labels for Pauli observables; each observable is distributed to the provers according to the scheme described in Definition~\ref{def:queries}). These questions are called ``EPR questions''. In addition, the verifier samples a query $(q_P,q_1,\ldots,q_r)$ as in $G$ and sends $q_i$ to the $i$-th prover (the question $q_P$ is ignored). This question is called ``$G$ question'' (each player gets two questions, one of each type).\\
	Each player replies with an answer to its EPR question, and an answer to its $G$ question. If $(A,\comp{A'})$ are the answers to the EPR questions, the verifier accepts if and only if $(A,A')$ would have been accepted in the EPR test. (Answers to $G$ questions are ignored.)  
		\item \emph{(Simulate $G$)} The verifier samples a query $(q_P,q_1,\ldots,q_r)$ as in $G$. He sends $q_P$ to the special prover and distributes $\comp{q_P}$ to the composite prover, where each question is formatted as an EPR question. In addition, the verifier sends the $G$ question $q_i$ to prover $i$, for $i\in\{1,\ldots,r\}$.\\
		Let $(A,\comp{A})$ be the answers to the EPR questions, and $(a_1,\ldots,a_r)$ the answers to the $G$ questions. The verifier accepts if and only if the tuple $(A+\comp{A},a_1,\ldots,a_r)$ is a valid answer tuple to the query $(q_P,q_1,\ldots,q_r)$ in $G$.
	\end{enumerate}    
\end{mdframed}
\end{center}
\begin{figure}[H]
\caption{Simulated Pauli Prover game $G'$.}
\label{fig:simpauli_check}
\end{figure}

\begin{theorem}
Let $G$ be an $S$-qubuit $(r+1)$-prover Single Pauli Prover game such that $r\geq 7$. Let $G'$ be the associated Simulated Pauli Prover game, as described in Figure~\ref{fig:simpauli_check}. Then the following hold:
\begin{itemize}
\item (Completeness) If there is a perfect strategy for the $(r+1)$ provers in $G$ such that the Pauli prover employs an honest Single-Prover Pauli strategy, then there is a perfect strategy for the $r$ provers in $G'$. 
\item (Soundness) For any $\eps\geq 0$ and any $r$-prover strategy that is accepted with probability at least $1-\eps$ in $G'$, there is an $(r+1)$-prover strategy in $G$ such that the Pauli prover employs an honest Single-Prover Pauli strategy, and that is accepted with probability at least $1-\poly[S;\eps]$ in $G$.
\end{itemize}
\end{theorem}

\begin{proof}
\tnote{very sketchy}
First we note that for any fixing of $(q_1,\ldots,q_r)$ the questions to the simulated Pauli prover in parts 1. and 2. of the test are indistinguishable. Using the analysis of part 1. that follows from Theorem~\ref{thm:pbt} as in~\cite{}, we deduce that, for any fixed $(q_1,\ldots,q_r)$ the simulated Pauli prover implements a honest $S$-qubit Pauli strategy. 

Next we observe that only the question $q_t$ to the special prover is changed, it must apply the same Pauli observable, as it should remain consistent with the composite prover. Therefore, the isometry and the Pauli observables do not depend on $q$. 

We define a strategy for $G$ as follows. The Pauli prover is given each of the $S$-qubit registers held by the simulated Pauli prover in $G'$. On reception a Pauli query, it simulates the actions of these provers. 

The remaining $r$ provers are given the remaining qubits of the $r$ other provers in $G'$. On reception a question in $G$, a prover samples a uniformly random (fake) Pauly query, and applies the associated observable. In general this observable may require $S$ extra qubits, so the prover uses $S$ qubits in the $\ket{0}$ state as a proxy. 

To complete the analysis it suffices to argue that the observables associated to $q_i$ act nearly as identity on the $S$-qubit Pauli registers. Informally, this follows because the $7$-qubit code corrects at least one error.
\end{proof}