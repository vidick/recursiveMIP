\section{The stabilizer encoding}
\label{sec:stabilizer}



\subsection{Analyzing $V_{UPC}$ with general strategies}

We now carry out the second stage of the analysis, where we analyze the value of the protocol where the VTM is $V_{UPC}$ instead of $V_H$. 

\paragraph{Provers} The set $\cP_r$ of provers that $V_{UPC}$ interacts with has size $\kappa_r = \kappa_{r-1} + 7$. What was originally prover $PV$ is now represented by provers $PV_1,\ldots,PV_7$. 

\subsection{Stabilizer check}

\begin{center}
\begin{mdframed}
    Input: $(1^n,w,G,V,M,r)$ \\
	\begin{enumerate}
		\item Do the multi qubit stabilizer test.
	\end{enumerate}    
\end{mdframed}
\end{center}
\begin{figure}[H]
\caption{Stabilizer Check}
\label{fig:stabilizer_check}
\end{figure}

\paragraph{The high level} The Stabilizer check is performed with the provers $PV_1,\ldots,PV_7$, and checks that they share a $7$-qubit stabilizer encoding of the registers $\what{\sO}, \what{\sK}_1,\what{\sK}_2, \what{\sC}, \what{\sV}, \what{\sM}, \what{\sE}_1, \what{\sE}_2$. 

% the prover $PV$ (which is supposed to play the role of the verifier of $UPC_{N,r-1}$) has the $\what{\sV}_{in}$ register initialized to the input $(G,V,M)$, and $\what{\sO} \what{\sV}_{work} \what{\sM} \what{\sK}_1 \what{\sK}_2$ set to zeroes. Furthermore, the subroutine will check that the $PV$ shares the maximally entangled state with $PP_i$ in the $\what{\sE}_{1,i} \what{\sE}_{2,i}$ registers.

%the $PP_i$ prover has the message register $\sM_{r-1,i}$ set to all zeroes.

\paragraph{Question types} 

\paragraph{Honest Input Check strategy}
A strategy $\strat = (\rho,\{M_i\})$ with shared state $\ket{\psi}_{\sC \sP \sR}$ is an \textbf{honest Stabilizer strategy} if the measurement operators of the provers $PV_1,\ldots,PV_7$ are all honest Pauli measurements, the registers $\sP_{V,1},\ldots,\sP_{V,7}$ contain a distributed encoding of the registers $\what{\sO}, \what{\sK}_1,\what{\sK}_2, \what{\sC}, \what{\sV}, \what{\sM}, \what{\sE}_1, \what{\sE}_2$.

\hnote{Probably better ways of saying all these things}


\begin{lemma}	
\label{lem:stabilizer_check}
\leavevmode
\begin{enumerate}
	\item (\textbf{Completeness}) An honest Stabilizer Check strategy passes the $\textbf{Stabilizer Check}$ subprotocol with probability $1$. 
	\item (\textbf{Soundness}) There exists a $\delta = \poly[N;\eps]$ such that all strategies $\strat$ that pass the $\textbf{Stabilizer Check}$ subprotocol with probability at least $1 - \eps$ are $\delta$-isometric to an honest Stabilizer strategy.
\end{enumerate}
\end{lemma}


\subsection{The other subroutines}
The \textbf{Gate check}, \textbf{Input check}, and \textbf{Output check} subroutines are exactly the same as \textbf{Gate2 check}, \textbf{Input2 check}, and \textbf{Output2 check} except whatever question was supposed to be sent to $PV$ is now sent simultaneously to $PV_1,\ldots,PV_r$. Answers that were meant to be from $PV$ are now computed by taking the XOR of all the answers from $PV_1,\ldots,PV_7$. 


\subsection{Prover simulation}

\paragraph{Honest Single-Prover Pauli strategy} For $r\geq 0$ we say that an $(r+1)$-prover strategy $(\ket{\psi},\{M_i\})$ for an extended nonlocal game is an ($S$-qubit) \emph{honest single-prover Pauli strategy} (implicitly, for the $t$-th prover, for some $t\in\{1,\ldots,r+1\}$) if the register of $\ket{\psi}$ that is associated with the $t$-th prover consists of $S$ qubits, and moreover on any question the answer bits returned by the $t$-th prover are obtained by performing a set of commuting observables (one for each answer bit), each of which can be expressed either as a tensor product of at most two $\sigma_X$ Pauli observables, or a tensor product of at most two single-qubit $\sigma_Z$ Pauli observables.



\begin{definition}
Call an extended nonlocal game an ($S$-qubit) \emph{Single Pauli Prover game} if the following holds. The game has $(r+1)$ provers. The first prover is called the ``Pauli prover''. In the game, queries take the form $Q=(q_P,q_1,\ldots,q_r)$, where the question $q_P$ to the Pauli prover is always a triple of the form $(W_1,W_2,W_3)$ where $W_1,W_2,W_3$ are labels for commuting  observables, each of which is a tensor product either of at most two single-qubit $\sigma_X$ Pauli observables, or at most two single-qubit $\sigma_Z$ Pauli observables. 
\end{definition}


Let $G$ be an $(r+1)$-prover Single Pauli Prover game such that $r\geq 7$. We define an $r$-prover \emph{Simulated Pauli Prover game} $G'$ by using $7$ out of the $r$ provers to simulate the Pauli prover in $G$ using a technique similar to the ``code-check'' test in~\cite{NV17} (see also~\cite{Ji}). We then show that any $r$-prover strategy in $G'$ can be ``decomposed'' into an $(r+1)$-prover strategy in $G$ with similar success probability. 

To achieve this we first define the kind of codes we need and introduce an appropriate multi-qubit test for constant-weight Pauli observables. 


\subsubsection{Background material}
\label{sec:codes}

\paragraph{Stabilizer codes}
We consider weakly self-dual  \emph{Calderbank-Shor-Steane (CSS)
  codes}~\cite{CalderbankShor96,Steane96}. Let $C$ be a classical $[k,k']$ linear error-correcting code over $\Fp_2$: $C$ is specified by a generator matrix $H \in \Fp_2^{k\times k'}$ and a parity check matrix $K\in \Fp_2^{(k-k')\times k}$ such that $C = \text{Im}(H)=\ker(K)$. We say that $C$ is weakly self-dual if the dual code $C^\perp$, with generator matrix $K^T$, is such that $C\subseteq C^\perp$; equivalently, $H^T H=0$. To any such  code $C$ we associate a subspace $\mathcal{C}$ of 
$(\C^2)^{\otimes k}$ that is the simultaneous $+1$ eigenspace of a set of stabilizers 
$\{S_{W,j}\}_{W\in\{X,Z\},j\in\{1,\ldots,k'\}}$ such that $S_{W,j}$ is
a tensor product of Pauli $\sigma_W$ observables over $\Fp_2$ in the locations indicated by the
$j$-th column of the generator matrix $H$, i.e.
\[ S_{W,j} = \sigma_W(H_{1j}) \otimes \sigma_W(H_{2j}) \otimes \cdots
  \otimes  \sigma_W(H_{kj}), \]
where $H_{ij}$ is the $(i,j)$-th entry of $G$.
The condition that $H^TH=0$ implies that all the $S_{W,j}$ commute, so that $\mathcal{C}$ is well-defined. 

\begin{example}[Quadratic residue code]
\label{ex:quad_res_code}
The $7$-qudit code is a weakly self-dual CSS code that has $k=7$, $k'=3$, and encodes one qudit over $\Fp_2$.
\end{example}

\paragraph{EPR test} 
By replacing the use of the CHSH test with the Magic Square test in the EPR test from~\cite{chao2016test} we obtain the following test and guarantees. 

\vspace{10pt}
\begin{center}
\begin{mdframed}
    Input: an integer $S$ and a set of $2S$ labels, $\{W_i|\, W\in\{X,Z\},\,i\in\{1,\ldots,S\}\}$.\\
		The verifier performs each of the following with equal probability:
		\begin{enumerate}
		\item Select $i\neq j\in\{1,\ldots,S\}$ and $W,W'\in\{X,Z\}$ uniformly at random. Send $(W_i,W'_j)$ to player A. Select two two-qubit Pauli observables $W^{(2)},W^{(3)}$ that mutually commute and commute with $WW''$. Send the (lexicographically ordered) triple $(W_iW'_j,W^{(2)},W^{(3)})$ to $B$. Receive bits $(a,a')$ from player A and $(b_1,b_2,b_3)$ from player $B$. Accept if and only if $aa'=b_1$.
		\item Select pairwise distinct $i,j,k,\ell\in\{1,\ldots,S\}$, a pair of questions $(q,q') \in \QMS\times\QMS$ in the Magic Square test, and a uniformly random $q''\in\QMS$. Send $(q,(i,j))$ to player A and the unordered pair $(q',(i,j))$ and $(q'',(k,\ell))$ to player B.\footnote{In all cases, if $q$ is of the form $(W,W')$, we label the question $(q,(i,j))$ as $(W_i,W'_j)$, to make it indistinguishable from a question in part 1. of the test.} Accept if and only if the players' answers associated with the query $(q,q')$ would be accepted in the Magic Square test. 
   \end{enumerate}    
\end{mdframed}
\end{center}
\begin{figure}[H]
\caption{$k$-qubit EPR test~\cite{chao2016test}}
\label{fig:epr_test}
\end{figure}

\paragraph{Honest EPR strategy} An \emph{honest $k$-qubit EPR strategy} $\mS = (\ket{\psi}, \{W_i\})$ is a two-player strategy that satisfies the following conditions. In the strategy the players share a state $\ket{\psi} = \ket{\Phi}_{\reg{A}'\reg{B}'} \otimes \ket{\psi}_{\reg{ABR}}$, where player A has registers $\reg{A}$ and $\reg{A}'$, player B has registers $\reg{B}$ and $\reg{B}'$, and register $\reg{R}$ is an arbitrary purifying register. The state $\ket{\Phi}_{\reg{A}'\reg{B}'}$ is an $S$-qubit maximally entangled state. When sent a question of the form $(W_i,W'_j)$, the player determines its two answer bits by measuring $\sigma_W$ on the $i$-th qubit and $\sigma_{W'}$ on the $j$-th qubit. The same holds if there are four $(W_i,W'_j,W''_k,W'''_\ell)$. If the question has the form $(W^{(1)},W^{(2)},W^{(3)})$ for three mutually commuting (at most) two-qubit Pauli observables, the player returns the three bits obtained by simultaneously  measuring the three observables. 

The following follows from the results in~\cite{chao2016test}. 

\begin{theorem}\label{thm:epr-test}
Let $S \geq 4$ be an integer.\footnote{We need $S\geq 4$ to be able to ask the ``confuse'' questions in part 2. of the test.} The $S$-qubit EPR test has the following guarantees. 
\begin{itemize}
\item (Complexity) Questions in the test are $O(\log S)$-bit long. Answers are $O(1)$-bit long. 
\item (Completeness) Any honest $S$-qubit EPR strategy succeeds with probability $1$ in the test. 
\item (Soundness) For any $\eps\geq 0$ there is a $\delta = \poly[S;\eps]$ such that any strategy that succeeds with probability at least $1-\eps$ in the test is $\delta$-isometric to a honest $S$-qubit EPR strategy. 
\end{itemize}
\end{theorem}

%\begin{theorem}[Pauli braiding test]\label{thm:pbt}
  %For any integer $S\geq 2$ there exists a two-player nonlocal game with $O(S)$-bit questions and $O(1)$-bit answers such
  %that any strategy
  %that has success probability at least $1-\eps$ for some
  %$\eps\geq 0$ must be based on observables $A(x),A(z),A(x,z)$ and an
  %entangled state $\ket{\psi}_{AB}$ such that up to local isometries
  %$$A(x) \approx_{\delta} \otimes_i \sigma_X^{x_i}, \qquad A(z) \approx_{\delta} \otimes_i \sigma_Z^{z_i},\qquad\text{and}
%\qquad \ket{\psi}_{AB} \approx_{\delta} \ket{\Phi}_{AB}^{\otimes n},$$
%where $\delta=\poly(\eps)$. 
%\end{theorem}

\subsubsection{Simulated Pauli Prover game}

For convenience, fix a self-dual CSS code with $k=7$, such as the code from Example~\ref{ex:quad_res_code}. Let $H$ be the generator matrix of the code. Let $G$ be an $S$-qubit $(r+1)$-prover Single Pauli Prover game such that $r\geq 7$. The associated Simulated Pauli prover game is an $r$-prover extended nonlocal game that is described in Figure~\ref{fig:simpauli_check}. Before the game starts, $7$ out of the $r$ provers are designated to be the ``simulated Pauli prover''. Once the game has started, the verifier splits the provers that constitute the simulated Pauli prover into two groups. One prover, indexed by $t\in\{1,\ldots, r\}$, is chosen
at random and called the ``special prover''. The remaining $6$ provers, indexed by $i_1,\ldots,i_6 \in \{1,\ldots,r\}\backslash\{t\}$, are
jointly called the ``composite prover''. In general a
prover is not told whether it is the special prover, or a composite prover. The description of the game involves notions of ``composite query'' and ``composite answer'' that are defined as follows.  

\begin{definition}[Composite queries and answers]\label{def:queries}
Let $W$ be an $S$-qubit Pauli observable.
\begin{enumerate}
\item The composite query associated with $W$, denoted $\comp{W}$, is
  obtained by sending each prover forming the composite prover the question $W$.
\item 
Given answers $(A_{i_j})_{i\in\{1,\ldots,6\}}$ from the $6$ provers
  forming the composite prover, the composite answer $\comp{A}$ is
  obtained by selecting a uniformly random vector $v$ in the column
  span of $H$ such that $v_7=1$, and computing the sum $\comp{A} =
  \sum_{j\in\{1,\ldots,6\}} v_j A_{i_j}$.
\end{enumerate}
\end{definition} 


\begin{center}
\begin{mdframed}
    Input: an $S$-qubit $(r+1)$-prover Single Pauli Prover game $G$ \\
		There are $r$ provers, $7$ of which have been designated as the ``simulated Pauli prover''. Among those $7$ provers, the verifier chooses one at random to be the ``special prover'' $P_t$, for $t\in\{1,\ldots,r\}$. The remaining $6$ provers, $P_{i_1},\ldots,P_{i_6}$, are jointly referred to as the ``composite prover''. The verifier performs either of the following with equal probability: 
	\begin{enumerate}
		\item \emph{(Stabilizer test)} The verifier generates a query $(q,q')$ in the $S$-qubit EPR test.  The verifier sends $q$ to the special prover, and distributes $\comp{q'}$ to the composite prover ($q'$ is in general a pair or triple of labels for Pauli observables; each observable is distributed to the provers according to the scheme described in Definition~\ref{def:queries}). These questions are called ``EPR questions''. In addition, the verifier samples a query $(q_P,q_1,\ldots,q_r)$ as in $G$ and sends $q_i$ to the $i$-th prover (the question $q_P$ is ignored). This question is called ``$G$ question'' (each player gets two questions, one of each type).\\
	Each player replies with an answer to its EPR question, and an answer to its $G$ question. If $(A,\comp{A'})$ are the answers to the EPR questions, the verifier accepts if and only if $(A,A')$ would have been accepted in the EPR test. (Answers to $G$ questions are ignored.)  
		\item \emph{(Simulate $G$)} The verifier samples a query $(q_P,q_1,\ldots,q_r)$ as in $G$. He sends $q_P$ to the special prover and distributes $\comp{q_P}$ to the composite prover, where each question is formatted as an EPR question. In addition, the verifier sends the $G$ question $q_i$ to prover $i$, for $i\in\{1,\ldots,r\}$.\\
		Let $(A,\comp{A})$ be the answers to the EPR questions, and $(a_1,\ldots,a_r)$ the answers to the $G$ questions. The verifier accepts if and only if the tuple $(A+\comp{A},a_1,\ldots,a_r)$ is a valid answer tuple to the query $(q_P,q_1,\ldots,q_r)$ in $G$.
	\end{enumerate}    
\end{mdframed}
\end{center}
\begin{figure}[H]
\caption{Simulated Pauli Prover game $G'$.}
\label{fig:simpauli_check}
\end{figure}

\begin{theorem}
Let $G$ be an $S$-qubuit $(r+1)$-prover Single Pauli Prover game such that $r\geq 7$, such that the distribution of questions in $G$ is a product distribution\tnote{I had to add this for consistency reasons. It's not strictly needed, but easier this way. Can we make it to hold, by just tweaking the actual distribution and using e.g. ``rejection sampling'' (i.e. dummy accepts for the verifier)?} Let $G'$ be the associated Simulated Pauli Prover game, as described in Figure~\ref{fig:simpauli_check}. Then the following hold:
\begin{itemize}
\item (Completeness) If there is a perfect strategy for the $(r+1)$ provers in $G$ such that the Pauli prover employs an honest Single-Prover Pauli strategy, then there is a perfect strategy for the $r$ provers in $G'$. 
\item (Soundness) For any $\eps\geq 0$ and any $r$-prover strategy that is accepted with probability at least $1-\eps$ in $G'$, there is an $(r+1)$-prover strategy in $G$ such that the Pauli prover employs an honest Single-Prover Pauli strategy, and that is accepted with probability at least $1-\poly[S;\eps]$ in $G$.
\end{itemize}
\end{theorem}

\begin{proof}\tnote{still sketchy, but maybe detailed enough for a first pass}
Completeness of the test is clear. We show soundness. 
Fix a strategy $(\ket{\psi},\{M_i\})$ for the $r$ provers in $G'$ that has success probability at least $1-\eps$, for some $\eps\geq 0$. 



The following claim derives consequences of the strategy's success in the first part of the game, the stabilizer test. 

\begin{claim}\label{claim:stab-epr}
Let $t$ be the index of the special prover, and $\mH_t$ the Hilbert space that the prover's measurements act on. There exists an isometry $V_t :\mH_t \to (\C^2)^{\otimes S} \otimes \mH'_t$ such that the following hold. For any EPR question $W$, let $\{\sigma_W^a\}$ denote the associated Pauli projective measurement on $(\C^2)^{\otimes S}$. Then for any $G$ question $q$ there is a projective measurement  $\{\hat{M}_{W,q}^A\}$ on $ (\C^2)^{\otimes S} \otimes \mH'_t$ such that
\[ \sum_{A,a} \big\|\big(V_t M_{W,q}^{A,a} -  \hat{M}_{W,q}^A \sigma_W^a V_t \big)\ket{\psi} \big\|^2 \,=\, \poly[N;\eps]\;.\]
\end{claim}

In words, the claim shows that the special prover's measurement upon reception of an EPR question $W$ and a $G$ question $q$ can be simulated by (i) applying the isometry; (ii) performing the ``honest'' EPR measurement $\sigma_W$ on the first $S$ qubits; (iii) performing an arbitrary measurement, depending on both $W$ and $q$, on the post-measurement state. 

\begin{proof}
Fix an arbitrary $G$ query $Q=(q_1,\ldots,q_r)$ in part 1. of the test. Conditioned on this question being sent, the players' strategy (depending on $Q$) to determine their answers to the EPR questions has success probability $1-\poly[N;\eps]$ in the stabilizer test. 

From the players' strategy in $G$ (restricted to the $G$ query $Q$) we construct a strategy for the EPR test as follows. For player A we take the special prover's strategy. For player B we combine the strategies of the six provers that make the composite prover (including the post-processing involved in computing the composite answer $\comp{A'}$).

The resulting two-player strategy succeeds in the EPR test with success probability $1-\poly[N;\eps]$. Applying the soundness analysis of the EPR test given in Theorem~\ref{thm:epr-test} it follows that there is an isometry $V_t$ as described in the claim, such that under the isometry the special prover's measurement associated with the answer to the EPR question (where the answer to the $G$ question is marginalized) is $\poly[N;\eps]$-close to the ``honest'' Pauli observable on the $S$ qubits identified by the isometry. 

As a consequence, if $\hat{M}_{W,q}^{A,a}$ (where $q=q_t$) is the image of $M_{W,q}^{A,a}$ under the isometry, we have that $\sum_A \hat{M}_{W,q}^{A,a} \approx \sigma_W^a$. Using that the measurement is projective, multiplying this equation by $\hat{M}_{W,q}^{A}$ on both sides gives $\hat{M}_{W,q}^{A,a} \approx \hat{M}_{W,q}^{A}\sigma_W^a$. 

To complete the proof of the claim it remains to argue that the same isometry $V_t$ can be used irrespective of the choice of the $G$ query $Q$. For this we first apply the preceding argument to all provers that constitute the simulated prover, to obtain isometries $V_i$, each depending on the provers' $G$ question $q_i$. For the fixed question $Q$, success in the stabilizer test implies that for any Pauli observable $\sigma_W$ that can be asked as part of an EPR question, 
\[\big\| \big(\bigotimes_i \sigma_W - \Id \big) \big(\bigotimes_i V_i\big) \ket{\psi}\big\|^2 \,=\,\poly[N;\eps]\;.\] 
The same equation holds, with another isometry $V'_t$, in case the special prover's $G$ question is changed from $q_t$ to $q'_t$; it follows that $V_t \ket{\psi} \approx V'_t \ket{\psi}$, so we may use the isometry associated with an arbitrary $G$ question to the special prover as the defining isometry for all questions. \end{proof}

With a well-defined isometry local to each prover, independent of the prover's $G$ question $q_i$, and an associated set of Pauli observables, we define a strategy for $G$ as follows:
\begin{itemize}
\item The Pauli prover is given each of the $S$-qubit registers held by the $7$ provers that constitute the simulated Pauli prover in $G'$ (after application of the isometry). On reception of an EPR question in $G$ (recall that by assumptions all questions to the Pauli prover in $G$ are EPR questions), the prover reproduces the actions of the simulated Pauli prover in $G$, i.e. it applied the logical operator associated with the EPR question on the $7S$-qubit encoded states. It returns the sum of the simulated Pauli prover's answers, i.e. the decoded outcomes of the logical measurements, as its three-bit answer.
\item Each of the remaining $r$ provers in $G$ is given the remaining qubits of the corresponding prover in $G'$, i.e. the qubits that are not part of the $S$ qubits singled out by the isometry. These qubits are replaced by a totally mixed  state (for each prover, uncorrelated with the others), so that the prover's observables from $G'$ have a well-defined action on the prover's register in $G$. On reception of a question in $G$, the prover samples a uniformly random EPR question, and performs the same measurement as the prover would in the game $G'$. It ignores the answer to the EPR question, and returns the answer to the $G$ question. 
\end{itemize} 
We argue that the success probability of this strategy in $G$ is polynomially related to the success probability of the strategy in part 2. of $G'$. The following claim allows us to ``simplify'' each provers' observables associated with the answer to a $G$ question. 

\begin{claim}\label{claim:stab-g}
Fix a query $(q_P,q_1,\ldots,q_r)$ in $G$. Let $(W,i)$, where $i\in\{1,\ldots,S\}$ and $W\in\{X,Z\}$, be an EPR question that consists of a single-qubit Pauli operator. Write $\{W_i\}$ for the associated observable, leaving the prover index implicit. (By Claim~\ref{claim:stab-epr} this observable is well-defined, independently of the $G$ question.) Let $W'$ be (any) EPR question. Let $\{M_{W',q_t}^{A,a}\}$ be the projective measurement applied by the special prover on question $(W',q_t)$. Let $\tilde{M}_{W',q_t}^{A,a} = W_i M_{W',q_t}^{A'a} W_i$. Let $W''$ be an EPR question that contains a measurement of $(W,i)$, and $\{M_{W'',q_i}^{A,a}\}$ the $i$-th prover projective measurement in $G$. Then the following holds: 
\[ \sum_{\substack{(A_1,\ldots,A_r)\\(a_1,\ldots,a_r)}}\,\Big|\bra{\psi}\Big( \bigotimes_{\substack{i=1\\i\neq t}}^r M_{W'',q_i}^{A_i,a_i} \Big)\otimes\Big(M_{W',q_t}^{A_t,a_t}  - \tilde{M}_{W',q_t}^{A_t,a_t}  \Big)  \ket{\psi}\Big| \,=\, \poly[N;\eps]\;.\]

\end{claim}

\begin{proof}
By part 1. of the test and Claim~\ref{claim:stab-epr} it follows that the state $\ket{\psi}$ is (approximately) stabilized by $\otimes W_i$, where the tensor product is taken over all provers that form the simulated prover in $G'$. Moreover, for any $i\neq t$, by Claim~\ref{claim:stab-epr} the measurement $M_{W'',q_i}^{A_i,a_i}$ approximately commutes with $W_i$, since $W''$ contains a measurement of $W_i$ as one of the three observables that define it. The claim follows. 
\end{proof}

Applying Claim~\ref{claim:stab-g} $(2N)$ times in sequence, for each choice of a single-qubit Pauli observable $(W,i)$, we deduce that the distribution on answers in $G'$ that is obtained by the strategy where the special prover applies measurement operators $M_{W',q_t}^{A,a}$, or where it applies the operators
\[ \tilde{M}_{W',q_t}^{A,a} \,=\, \Es{W\in\{I,X,Z,XZ\}^N} \, \sigma_W \big(\hat{M}_{W',q_t}^{A,a}\big) \sigma_W^\dagger \;,\]
differ by at most $\poly[N;\eps]$. To conclude, we observe that the action of $\{\tilde{M}_{W',q_t}^{A,a}$ on the state $\ket{\psi}$, or on the state $\ket{\psi}$ in which the $S$ qubits identified by the isometry for the special prover have been replaced by $S$ qubits in the totally mixed state, is identical. 
\end{proof}