\section{Analysis of the Pauli check}



%%%%%%%%%%%%%%%%%%%%%%%%%%		PAULI CHECK		%%%%%%%%%%%%%%%%%%%%%%%%%%%%%%%

\subsection{Pauli check}
\label{sec:pauli_check}

The \textsc{Pauli Check} subroutine describes an interaction between the $r$-th level verifier and the provers $\mathcal{P}_r$. In the subroutine, the verifier interacts with two players, taken from the set $\mathcal{P}_r$. The first player is always $PA$, and called ``player A'' in the subroutine. The second player is always prover $PV$, and called ``player B''. For the rest of the subroutine, the verifier will only interact with players A and B. 
\hnote{The Pauli Check is only performed between $PA$ and $PV$ now, because we're dealing with simcom games.}

%Note that player A is always told which prover plays the role of player B.  
%provers $PA_1$ and $PV$. In the description of the check we refer to the former as ``player A'' and to the latter as ``player B''. 

In the subroutine, the verifier performs one of several tests, chosen uniformly at random. The tests are summarized in Figure~\ref{fig:pauli_check} in an ``operational'' language. Formally, each test should be put in the form of actions for the VTM, following the template introduced in Figure~\ref{fig:check_structure}. 

\vspace{10pt}
\begin{center}
\begin{mdframed}
    Input: $(1^n,w,r,G,V,M)$ \\
    \\
	Let player A denote prover $PA$.
    Choose a random prover $P^*$ from the set of non-ancilla provers $\mathcal{R}_r$, and let player B denote prover $P^*$. Send player A the label $P^*$. \\
    \\
    Perform one of the tests uniformly at random. Each test should be interpreted as a three-phase procedure, as described in Figure~\ref{fig:check_structure}. The \textsc{ClockMeasurement} phase involves a measurement on the clock register $\reg{C}_p$ of dimension $2(N+1)(\kappa(r-1)+1)$, with computational basis labeled by 
    \[
    (t,t')\in \{0,\ldots,2N\} \times \{0,\ldots,2N\}\; .
    \]
    
	\begin{enumerate}

\item \emph{(Propagation check)} Let $\reg{C}_{p,1}$ and $\reg{C}_{p,2}$ denote the two registers of $\reg{C}_p$. Execute each of the following with equal probability:
\begin{enumerate}
\item Measure register $\reg{C}_{p,2}$ to obtain an outcome $t'\in\{0,\ldots,2N\}$. If $t'\neq 0$ then accept. Otherwise, execute $\prop(2N,\reg{C}_{p,1},\{CW_1,\ldots,CW_{2N}\})$ between the verifier and B.
\item Execute $\prop(2N,\reg{C}_{p,2},\{W_1,\ldots,W_{2N}\})$ between the verifier and B.
\end{enumerate}

 \item \emph{(EPR test)} Measure the first register of $\reg{C}_{p}$ in the computational basis to obtain an outcome $t\in\{0,\ldots,2N\}$. 
%Select $t'\in\{1,\ldots,2N\}$ uniformly at random. Measure the second register of $\reg{C}_{p}$ using a basis containing $\{\ket{+_{0,t'}},\ket{-_{0,t'}}\}$. Accept if neither of these two outcomes is obtained. 
Generate questions $(q_A,q_B)$ as in the $2N$-qubit EPR test with A and B. Send $q_A$ to Alice, and $q_B$ to B. Accept if $t\neq 0$ or if the Pauli question to B is of the form $CW_t$. Otherwise, accept if and only if the answers to the EPR questions are valid.
%\tnote{Intended goal: Checks that whenever A or B is sent an EPR question, it applies the corresponding Pauli observable, independently of any other question asked. A consequence of this test and the form of questions to player B is that we can write the observables it applies as a product of an observable for the EPR question, and an observable for the Pauli question, where each observable only depends on the corresponding question (and not on the pair of questions).}

\item \emph{(CTL check)} Measure the first register of $\reg{C}_{p}$ in a basis containing $\{\ket{+_t},\ket{-_t}\}$ to obtain an outcome $c$ where $c\in\{0,1,2\}$. If $c=2$ then accept. Send player A the EPR question $X_{t}$, and send player $B$ the Pauli question $W_t$. Accept if and only if, either A's answer is $0$ and $c=1$, or A's answer is $1$ and $c$ matches B's answer to the Pauli question. %If $t'>0$, accept if $t\neq t'$. Send player A the EPR question $X_t$ and accept if the answer is not $1$. Send player B the Pauli question $W'_t$ and accept if and only if the outcome matches $c$.

\item{\emph{(Pauli relation check)}} Measure both  registers of $\reg{C}_p$  in the computational basis to obtain outcomes $t,t'\in\{0,\ldots,2N\}$. Accept if $(t,t')\neq (2N,0)$. Select $u\in\{1,\ldots,N\}$ uniformly at random. Send A the EPR question $X_u$, and B the EPR question $Z_u$. Do either of the following with equal probability:
\begin{enumerate}
\item Send B the Pauli question $W_{2u}$. Accept if and only if the product of the players' three answers is $1$.
\item Send B the Pauli question $W_{2u-1}$. Accept if and only if A's answer to the EPR question matches B's answer to the Pauli question. 
\end{enumerate}  
\end{enumerate}  
\end{mdframed}
\end{center}
\begin{figure}[H]
\caption{Pauli Check}
\label{fig:pauli_check}
\end{figure}


\paragraph{Question types}
There are three types of questions: ``EPR questions'', ``Pauli questions'' and ``CTL-Pauli questions''. 
\begin{itemize}
\item \emph{EPR questions} range over the question set of the $2N$-qubit EPR test \tnote{this needs to be defined somewhere}. They are of the form $X'_i$ or $Z'_j$, for $i,j\in\{1,\ldots,2N\}$, or $(V'_i,W'_j)$ for $V,W\in\{X,Z\}$ and $i\neq j \in \{1,\ldots,2N\}$. \tnote{In fact there are additional question types involved, but these are the more relevant ones} When player A is sent an EPR question, it is additionally sent the identity of player B (i.e. which prover in $\mathcal{R}_r$ it corresponds to).
\item \emph{Pauli questions} range over the set of subsets 
$$\{W_1(w_1),W_2(w_2),W_3(w_3)\}\,\subseteq\,\{X(a),Z(b)\,|\; a,b\in\{0,1\}^N\}$$
 of cardinality at most $3$, and such that all (at most) three Pauli operators $W_i(w_i)$ in the set mutually commute, and all strings $w_i$ have Hamming weight at most $3$.
\item \emph{CTL-Pauli questions} range over the set $\{CX_i,CZ_j|i,j\in\{1,\ldots,N\}\}$.
\end{itemize}
Player A is always sent an EPR question and the identity of player B. Player B is always sent a pair consisting of an EPR question and a Pauli, or CTL-Pauli question. In the test as described in Figure~\ref{fig:pauli_check} Pauli questions consist of a single Pauli operator. In fact, the Pauli question is always embedded in a triple of commuting Pauli operators of weight at most $3$, where the other two operators are chosen uniformly among pairs that mutually commute and commute with the ``real'' question. In addition, even if the test only calls for a Pauli question, an EPR question is asked and chosen at random. (For a CTL-Pauli question of the form $CX_i$ (resp. $CZ_j$), the EPR question can be any question not involving the index $i$ (resp. $n+j$).) In all cases, only the player's answer to the ``real'' question is taken into account in the verifier's decision. 

\paragraph{Honest Pauli Check strategy}
An \emph{honest Pauli Check strategy} is a strategy $\mathcal{S} = (\ket{\psi},\{ M_P \})$ that satisfies the following conditions. The state $\ket{\psi}$ is a pure state on registers $\sC \sP_1 \cdots \sP_{\kappa(r)} \sR$, where $\sC = \sC_p \sC_{mip}$ is a clock register given to the $r$-th level verifier, $\sP_1 \cdots \sP_{\kappa(r)}$ are registers distributed among the provers in $\mathcal{P}_r$, and $\sR$ is a purifying register. For all provers $P \in \mathcal{P}_r$, the register $\sP_P$ is given to $P$.\footnote{Here we label the prover registers in two ways. To refer to prover $P$'s registers, we use $\sP_P$. If we are simply iterating over them, then we use an integer index such as $\sP_i$.}  For each non-ancilla prover $P \in \mathcal{R}_r$, the register $\sP_P$ includes two registers $\sB'_P \sB_P$, where $\sB'_P$ consists of $2N$ qubits and $\sB_P$ consists of $N$ qubits. The register $\sP_{PA}$ (corresponding to ancilla prover $PA$) includes registers $\{ \sA_P' \}_{P \in \mathcal{R}_r}$, where $\sA_P'$ consists of $2N$ qubits. We let $\sZ$ denote the union of the qubits in $\sP_1 \cdots \sP_{\kappa(r)}$ that are not part of $\{ \sA_P' \sB_P' \sB_P \}$. 
\hnote{At some point we should make a picture of all the provers with all the registers.}
\tnote{the description should be for a designated prover $P^*$; right now this is not specified correctly}

The state $\ket{\psi}$ has the following form:
\begin{equation}
	\ket{\psi} = \frac{1}{2N+1} \sum_{ t,t' \in \set{0,\ldots,2N} } \ket{t,t'}_{\sC_p} \otimes \ket{\psi_{t,t'}}_{\sC_{mip} \sP \sR}\;,
	\label{eq:honest-psi}
\end{equation}
where 
\[
	\ket{\psi_{0,0}}_{\sC_{mip} \sP \sR} = \Paren{ \bigotimes_{P \in \mathcal{R}_r} \ket{\Phi}_{\sA_P' \sB_P'}} \otimes \ket{\varphi_0}_{\sC_{mip} \sB_{PV} \cdots \sB_{PP_{\kappa(r-1)}} \sZ \sR}\;,
\]
for some state $\ket{\varphi_0}$. Here, the state $\ket{\Phi}_{\sA_P' \sB_P'}$ is a tensor product of $2N$ EPR pairs across registers $\sA_P'$ and $\sB_P'$. For $t\geq 0$ and 
$P \in \mathcal{R}_r$, define inductively
\begin{align}
	\ket{\psi_{0,t+1}} =  CW_t \ket{\psi_{0,t}} \;,
\end{align}
and for $t' \geq 0$ define 
\begin{align}
	\ket{\psi_{t'+1,t}} =  W_{t'} \ket{\psi_{t',t}} \;,
\end{align}
where if $t$ or $t'$ is even (resp. odd), $CW_t$ or $W_{t'}$ is a controlled-$X$ or $X$ (resp. controlled-$Z$ or $Z$) gate whose control qubit is the $\sB_P'$ part of the $t$-th EPR pair of $\ket{\Phi}_{\sA_P' \sB_P'}$, and whose target qubit is the $\lceil t/2\rceil$-th qubit of $\sB_P$. 

\tnote{is something like this ok?}\tnote{currently the test only gives one Pauli, not two}Moreover, there are $2N$ observables $\{W_i|W\in\{X,Z\},i\in\{1,\ldots,N\}$ acting on register $\reg{B}_{P^*}$ such that the $W_i$ satisfy the Pauli (anti-)commutation relations, and moreover whenever prover $P^*$ is sent a question labeled as $(W_i,W'_j)$ for $W,W'\in\{X,Z\}$ and $i,j\in\{1,\ldots,N\}$ such that $W_i$ and $W'_j$ commute, it returns the two outcomes obtained by performing the corresponding measurements on register $\reg{B}_P$ of state $\ket{\psi}$. 
\tnote{Here I am formulating this as if the strategy is Pauli on the whole state. Then, the soundness should say that a successful strategy is isometric to this one, \emph{on the state $\ket{\psi_{0,0}}$}, but not on on $\ket{\psi}$} 



\paragraph{Guarantees}


\begin{lemma}[Completeness and soundness of the Pauli Check]
\label{lem:pauli_check}
\leavevmode
\begin{enumerate}
	\item (\textbf{Completeness}) Any honest Pauli Check strategy passes the Pauli Check  with probability $1$.
	\item (\textbf{Soundness}) For any $\eps\geq 0$ there is a $\delta = \poly[N;\eps]$ such that any strategy $\mathcal{S}$ that passes the Pauli Check with probability at least $1 - \eps$ is $\delta$-isometric to an honest Pauli Check strategy on the state $\ket{\psi_{0,0}}$ that is obtained by projecting register $\reg{C}_p$ on $\ket{0,0}$. \tnote{is this formulation ok}
\end{enumerate}
\end{lemma}

The completeness part of the lemma is shown in Lemma~\ref{lem:paulicheck-compleness}. The soundness is given in Lemma~\ref{lem:paulicheck-soundness}. 





\subsection{The EPR test}

Let $k\geq 4$ be an integer. 
The EPR test is an elementary test that aims to verify that two players A and B share $k$ EPR pairs, on which they measure one or two commuting single-qubit Pauli operators when asked to do so. 

The test can be built from any test for (at least) one EPR pair, such as the Magic Square test. We recall the properties of the test in a form that is useful for us. 

\begin{theorem}[Magic Square test, Theorem 5.9 in~\cite{coladangelo2017robust}]\label{thm:ms-rigid}
There exists a two-player test $\MS$ with the following properties:
\begin{enumerate}
\item Queries $(q_A,q_B)$ in the test are drawn uniformly from $\QMS \times \QMS$, where $\QMS$ is a finite set that contains a subset of $4$ elements designated $(W,W')$ for $W,W'\in\{X,Z\}$. 
\item Each player always replies with $2$ bits in $\{\pm 1\}^2$;
\item (Completeness) There is a strategy for the players in the test that consists of sharing two EPR pairs, and in which the answer to each question is determined by performing a Pauli measurement on the player's half-EPR pairs. In particular, answers to questions of the form $(W,W')$ for $W,W'\in\{X,Z\}$ are determined by applying the corresponding Pauli observable on each of a player's two qubits. 
\item For any $\eps\geq 0$ there is a $\delta = O(\sqrt{\eps})$ such that
   any strategy with success probability at least $1-\eps$ in the test is $\delta$-isometric to the honest strategy. 
\end{enumerate}
\end{theorem}

By replacing the use of the CHSH test with the Magic Square test in the EPR test from~\cite{chao2016test} we obtain the following test and guarantees. 

\vspace{10pt}
\begin{center}
\begin{mdframed}
    Input: an integer $k$ and a set of $2k$ labels, $\{W_i|\, W\in\{X,Z\},\,i\in\{1,\ldots,k\}\}$.\\
		The verifier perfors each of the following with equal probability:
		\begin{enumerate}
		\item Select $i\neq j\in\{1,\ldots,k\}$ and $W,W'\in\{X,Z\}$ uniformly at random. Send $(W_i,W'_j)$ to both players. Receive bits $a,a'$ from player A and $b,b'$ from player $B$. Accept if and only if $a=a'$ and $b=b'$.
		\item Select pairwise distinct $i,j,k,\ell\in\{1,\ldots,k\}$, a pair of questions $(Q,Q') \in \QMS\times\QMS$ in the Magic Square test, and a uniformly random $Q''\in\QMS$. Send $(Q,(i,j))$ to player A and the unordered pair $(Q',(i,j))$ and $(Q'',(k,\ell))$ to player B.\footnote{In all cases, if $Q$ is of the form $(W,W')$, we label the question $(Q,(i,j))$ as $(W_i,W'_j)$, to make it indistinguishable from a question in part 1. of the test.} Accept if and only if the players' answers associated with the query $(Q,Q')$ would be accepted in the Magic Square test. 
   \end{enumerate}    
\end{mdframed}
\end{center}
\begin{figure}[H]
\caption{$k$-qubit EPR test~\cite{chao2016test}}
\label{fig:epr_test}
\end{figure}

\paragraph{Honest EPR strategy} An \emph{honest $k$-qubit EPR strategy} $\mS = (\ket{\psi}, \{W_i\})$ is a two-player strategy that satisfies the following conditions. In the strategy the players share a state $\ket{\psi} = \ket{\Phi}_{\reg{A}'\reg{B}'} \otimes \ket{\psi}_{\reg{ABR}}$, where player A has registers $\reg{A}$ and $\reg{A}'$, player B has registers $\reg{B}$ and $\reg{B}'$, and register $\reg{R}$ is an arbitrary purifying register. The state $\ket{\Phi}_{\reg{A}'\reg{B}'}$ is a $k$-qubit maximally entangled state. Furthermore, each player has a set of $2k$ observables $W_i$, for $W\in \{X,Z\}$ and $i\in\{1,\ldots,k\}$, acting on its respective primed register, such that the observables satisfy the $k$-qubit Pauli (anti)-commutation relations, and the player determines its answer to any question of the form $(W_i,W'_j)$ by measuring the commuting observables $W_i$ and $W'_j$ and returning both outcomes. 

The following follows from the results in~\cite{chao2016test}. 

\begin{theorem}\label{thm:epr-test}
Let $k \geq 4$ be an integer.\footnote{We need $k\geq 4$ to be able to ask the ``confuse'' questions in part 2. of the test.} The $k$-qubit EPR test has the following guarantees. 
\begin{itemize}
\item (Completeness) Any honest $k$-qubit EPR strategy succeeds with probability $1$ in the test. 
\item (Soundness) For any $\eps\geq 0$ there is a $\delta = \poly[k;\eps]$ such that any strategy that succeeds with probability at least $1-\eps$ in the test is $\delta$-isometric to a honest $k$-qubit EPR strategy. 
\end{itemize}
\end{theorem}

\subsection{Completeness of the Pauli check}

The following lemma states the completeness property of the Pauli check. 

\begin{lemma}\label{lem:paulicheck-compleness}
Let $\mS$ be an honest Pauli Check strategy. Then the strategy is accepted with probability $1$ in the Pauli check.
\end{lemma}

\begin{proof}
Follows from the definition. 
\end{proof}

\subsection{Soundness of the Pauli check}

\begin{lemma}\label{lem:paulicheck-soundness}
 For any $\eps\geq 0$ there is a $\delta = \poly[N;\eps]$ such that any strategy $\mathcal{S}$ that passes the Pauli Check with probability at least $1 - \eps$ is $\delta$-isometric to an honest Pauli Check strategy on the state $\ket{\psi_{0,0}}$ that is obtained by projecting register $\reg{C}_p$ on $\ket{0,0}$. 
\end{lemma}

The lemma is proved using a sequence of claims that analyze each of the sub-tests that constitute the Pauli check. We start with the propagation check. 

\begin{claim}\label{claim:pauli-prop}
Let $\mathcal{S}$ be a strategy that passes the propagation check with probability at least $1 - \eps_1$. Then $\mathcal{S}$ is $\delta_1$-isometric to a strategy $\mathcal{S}_1$ such that the shared state takes the form
\[ \ket{\psi}_{\reg{C}_p \reg{PR}} \,=\, \frac{1}{2N+1} \sum_{t,t'=0}^{2N} \ket{t,t'}_{\reg{C}_p} W_{t'} CW_t \ket{ \psi_{0,0}}_{\reg{ZR}}\;, \]
for some observables $CW_t$ and $W_{t'}$ acting on $\reg{P}_B$. Note that here, as always, we use the label $\reg{Z}$ to denote all unspecified registers; in this case, $\reg{Z} = \reg{C}_{mip}\reg{P}$. 
 Moreover, when B is sent a Pauli question labeled $CW_t$ or $W_t$ then he measures the corresponding observable to determine its answer to that question (independently of the accompanying EPR question). \tnote{need a way to say that, for the $CW_t$, this is only conditioned on $t'=0$.}
\end{claim}

\begin{proof}
The claim follows from two successive applications of Lemma~\ref{lem:prop_check_0}.
\end{proof}


Next we analyze the consequences of the Pauli test. 

\begin{claim}\label{claim:pauli-epr}
Let $\mathcal{S}_1$ be a strategy satisfying the conclusion of Claim~\ref{claim:pauli-prop}. Suppose further that $\mS_1$ passes the EPR test with probability at least $1 - \eps_2$. Then $\mathcal{S}_1$ is $\delta_2$-isometric to a strategy $\mathcal{S}_2$, for some $\delta_2 = \poly[N;\eps_2]$, such that the following hold:
\begin{enumerate}
\item B always determines its answer to a question using two commuting binary observables, one associated with the EPR question and depending only on it, and the other associated with the Pauli question and depending only on it.  
\item The sub-strategy of $\mathcal{S}_2$ that consists of the shared entangled state, conditioned on $t=0$ in $\sC_{p,1}$, and the players' observables associated to Pauli questions, is an honest $2N$-qubit EPR strategy. In particular, the state $\ket{\psi_{0,0}}_{\reg{ZR}}$ takes the form 
\[ \ket{\psi_{0,0}}_{\reg{ZR}} \,=\, \ket{\Phi}_{\reg{A}'\reg{B}'} \ket{\varphi_0}_{\reg{ABZR}}\;,\]
where registers $\reg{A}\reg{A}'$ are in the possession of player A, and registers $\reg{B}\reg{B}'$ in the possession of player B. 
\item Conditioned on $\reg{C}_{p,1}$ being in state $t=0$, the observable associated to player B's answer to a Pauli question of the form $W_{t'}$ acts trivially on $\reg{B}'$. %The observable associated with $CW_t$ acts trivially on all qubits in $\reg{B}'$ but the $t$-th qubit.
\end{enumerate}  
\end{claim}

\begin{proof}
First we observe that for any EPR question to player $B$, player A and player B are required to successfully play a copy of the $2N$-qubit EPR test, where questions to player A do not depend on the Pauli question to B. Using Theorem~\ref{thm:epr-test}, this implies that the observable associated with player B's answer to the EPR question does not depend on the Pauli question. Since the converse property follows from Claim~\ref{claim:pauli-prop}, this shows the first item. The second item follows directly from Theorem~\ref{thm:epr-test}. The last item follows from the first, since an observable commutes with all Pauli observables on a qubit if and only if it acts trivially on that qubit.   
\end{proof}

%\begin{claim}\label{claim:pauli-propagation}
%Let $\mathcal{S}_1$ be a strategy satisfying the conclusion of Claim~\ref{claim:pauli-epr}. Suppose further that $\mS_1$ passes the propagation test with probability at least $1-\eps_2$. Then $\mathcal{S}_1$ is $\delta_2$-isometric to an honest Propagation Check strategy $\mathcal{S}_2$ with the following properties:
%\begin{enumerate}
%\item The clock register is the register $\sC_p$. 
%\item The initial state for the computation takes the form 
%\begin{equation}\label{eq:pauli-prop-1}
%\ket{\Phi}_{\sA'_A\sB'_B} \ket{\theta}_{\sA\sB\sZ\sR}\;,
%\end{equation}
%where $\ket{\Phi}_{\sA'_A\sB'_B}$ is the $2N$-qubit maximally entangled state associated with the honest EPR strategy, $\sZ$ denotes a set of registers held by the provers, and $\sR$ a purifying register.
%\end{enumerate}
%\end{claim}
%
%\begin{proof}
%The fact that $\mS_1$ is isometric to an honest Propagation Check strategy follows from the assumption that  $\mS_1$ passes the propagation test with probability $1-\eps_2$, and the soundness statement in Lemma~\ref{lem:prop_check} \tnote{the Lemma should be adapted a little for this to work out.}. The form of the initial state follows from the assumption that $\mS_1$ is an honest EPR strategy, so we may choose the isometry for the Propagation check so that the provers' shared state has the form~\eqref{eq:pauli-prop-1}. 
%\end{proof}


\begin{claim}\label{claim:pauli-ctl}
Let $\mathcal{S}_2$ be a strategy satisfying the conclusion of Claim~\ref{claim:pauli-epr}. Suppose further that $\mS_2$ passes the CTL check with probability at least $1-\eps_3$. Then $\mS_2$ is $\delta_3$-close to a strategy in which the observable associated with a Pauli question $CW_t$, for $t\in\{1,\ldots,2N\}$, is the same observable as a CTL$-W_t$ obsersable, for $W_t$ the observable applied on Pauli question $W_t$, and where the control is the $t$-th qubit in $\sB'$. 
\end{claim}

\begin{proof}
Suppose the verifier obtains an outcome $c\neq 2$ for his measurement on $\reg{C}_{p,1}$. By assumption on $\mS_2$, the shared state is 
\[ \frac{1}{\sqrt{2(2N+1)}}\sum_{t'} \ket{t'}_{\reg{C}_{p,2}}\otimes \big(\ket{t}_{\reg{C}_{p,1}} G'_{t'} G_{t-1} \ket{\Phi}_{\reg{A}'\reg{B}'}  \ket{\varphi_0}_{\reg{ZR}} + (-1)^c  \ket{t}_{\reg{C}_{p,1}} \Id_{\reg{A}'} \otimes G_{t'} (CW_t)_{\reg{B}\reg{B}'} G_{t-1}\ket{\Phi}_{\reg{A}'\reg{B}'} \ket{\varphi_0}_{\reg{ZR}}\big)\;,\]
where $G_{t'} = W_{t'}\cdots W_1$ and $G_{t-1} =  CW_{t-1}\cdots CW_1$. 
In case player A reports the outcome $0$ for her EPR question $X_t$, the verifier's check implies that $CW_t$ acts as identity on $\reg{B}\reg{B}'$. In case player A reports the outcome $1$, the check implies that $CW_t$ acts as $G_{t'}^\dagger W_t G_{t'}$. Since this should hold for any $t'$, the claim follows. 
\end{proof}

\begin{claim}\label{claim:pauli-relation}
Let $\mathcal{S}_3$ be a strategy satisfying the conclusion of Claim~\ref{claim:pauli-ctl}. Suppose further that $\mS_3$ passes the Pauli relation check with probability at least $1-\eps_4$. Then the observables associated with player B's Pauli questions of the form $W_i$ satisfy the following conditions 
\begin{align}
\Es{i\in\{1,\ldots,N\}} \,\Es{a,b\in\{0,1\}^N}\,\big\| (X_i X(a) Z(b) -  X(a+e_i)Z(b) ) \ket{\varphi_0}\big\|^2\,=\,O(\delta_4)\;,\label{eq:x-com}\\
\Es{i\in\{1,\ldots,N\}} \,\Es{a,b\in\{0,1\}^N}\,\big\| (Z_i X(a) Z(b)- (-1)^{a_i} X(a)Z(b+e_i) ) \ket{\varphi_0}\big\|^2\,=\,O(\delta_4) \;,\label{eq:z-com}
\end{align}
for some $\delta_4 = \poly[N;\eps_3+\delta_3]$.
\end{claim}

\begin{proof}
Conditions~\eqref{eq:x-com} and~\eqref{eq:z-com} follow from the two parts of the test. Indeed, by assumption, conditioned on the verifier obtaining $t=2N$ and $t'=0$, the shared state is 
\[\ket{\psi} \,=\, \ket{2N,0}_{\reg{C}_p} \frac{1}{2^N}\sum_{a,b} \ket{a}_{\reg{A}'} \ket{b}_{\reg{B}'} X(a)Z(b)\ket{\psi_0}\;,\]
where we used the form of the $CW_t$ observables that follows from Claim~\ref{claim:pauli-ctl}.  
\end{proof}

To complete the proof of the soundness analysis we need the following lemma. 

\begin{lemma}\label{lem:gh}
Let $\delta\geq 0$, $\{X_i,Z_j:\, i,j\in\{1,\ldots,N\}\}$ observables on $\mathcal{H}_B$, and $\ket{\psi}_{BR}$ a state on $\mathcal{H}_B\otimes \cH_R$ such that  on average over $a,b\in\{0,1\}^N$ and $i,j\in\{1,\ldots,N\}$,
\begin{align}
\big\| (\Id_A \otimes X_i X(a) Z(b)\otimes \Id_R - \Id_A \otimes X(a+e_i)Z(b) \otimes \Id_R )\ket{\psi}_{ABR}\|^2\,\leq\,\delta\;,\label{eq:x-com-2}\\
\big\| (\Id_A \otimes Z_i X(a) Z(b)\otimes \Id_R - \Id_A \otimes (-1)^{a_i} X(a)Z(b+e_i) \otimes \Id_R )\ket{\psi}_{ABR}\big\|^2\,\leq\,\delta\;.\label{eq:z-com-2}
\end{align}
Then there is an isometry $V:\cH_B \to (\C^2)^{\otimes n} \otimes \cH_{B''}$ such that 
$$ \Es{i} \big\|  VX_i \ket{\psi} -  \sigma_X(e_i) V\ket{\psi} \big\| + \Es{j} \big\| VZ_j \ket{\psi} - \sigma_Z(e_j) V\ket{\psi} \big\| \,=\,O(\delta)\;,$$
where $\sigma_X(e_i)$ (resp. $\sigma_Z(e_j)$) acts on the $i$-th (resp. $j$-th) copy of $\C^2$ in the range of $V$. 
\end{lemma}

\begin{proof}
The relations~\eqref{eq:x-com} and~\eqref{eq:z-com} are sufficient to verify that $X(a)Z(b)$ is an approximate representation of the $n$-qubit Weyl-Heisenberg group (the matrix group generated by the Pauli $\sigma_X$ and $\sigma_Z$) Pauli group, in the sense of~\cite{gowers2017inverse}. The result then follows by an adaptation of the results of~\cite{gowers2017inverse} to arbitrary states, as stated in~\cite[Lemma 4.7]{coladangelo2017robust}.
\end{proof}

We are now ready to prove soundness of the Pauli check. 

\begin{proof}[Proof of Lemma~\ref{lem:paulicheck-soundness}]
Let $\mS$ be a strategy that succeeds in the test with probability at least $1-\eps$. Applying Claim~\ref{claim:pauli-prop}, Claim~\ref{claim:pauli-epr}, Claim~\ref{claim:pauli-ctl} and Claim~\ref{claim:pauli-relation} in sequence, the observables applied by B on Pauli questions satisfy the assumptions of Lemma~\ref{lem:gh}, when the state is conditioned on register $\reg{C}_p$ being in state $(t,t')=(0,0)$. The lemma follows. 
\end{proof}
