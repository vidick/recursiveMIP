\section{Analysis of the Pauli check}


\subsection{The EPR test}

Let $k\geq 4$ be an integer. 

The EPR test is an elementary test that aims to verify that two players A and B share $n$ EPR pairs, on which they measure one or two commuting single-qubit Pauli operators when asked to do so. 

The test can be built from any test for a single EPR pair, such as the Magic Square test. We recall the properties of the test in a form that is useful for us. 

\begin{theorem}[Magic Square test, Theorem 5.9 in~\cite{coladangelo2017robust}]\label{thm:ms-rigid}
There exists a two-player test $\MS$ with the following properties:
\begin{enumerate}
\item Queries $(q_A,q_B)$ in the game are drawn uniformly from $\QMS \times \QMS$, where $\QMS$ contains a subset of elements designated as $(W,W')$ for $W,W'\in\{X,Z\}$. 
\item Each player replies with $2$ bits in $\{\pm 1\}^2$;
\item (Completeness) There is a strategy for the players in the game that consists of sharing two EPR pairs and making Pauli measurements. 
\item For any $\eps\geq 0$ there is a $\delta = O(\sqrt{\eps})$ such that
  for any strategy with success probability at least $1-\eps$, the strategy is $\delta$-isometric to the honest strategy. 
\end{enumerate}
\end{theorem}

By replacing the use of the CHSH test with the Magic Square test in the EPR test from~\cite{reichardt} we obtain the following test and guarantees. 

\vspace{10pt}
\begin{center}
\begin{mdframed}
    Input: an integer $k$ and a set of $2k$ labels, $\{W_i|\, W\in\{X,Z\},\,i\in\{1,\ldots,k\}\}$.\\
		The verifier perfors each of the following with equal probability:
		\begin{enumerate}
		\item Select $i\neq j\in\{1,\ldots,k\}$ and $W,W'\in\{X,Z\}$ uniformly at random. Send $((W,W'),(i,j))$ to both players. Receive bits $a,a'$ from player A and $b,b'$ from player $B$. Accept if and only if $a=a'$ and $b=b'$.
		\item Select pairwise distinct $i,j,k,\ell\in\{1,\ldots,k\}$, a pair of questions $(Q,Q') \in \QMS\times\QMS$ in the Magic Square test, a uniform $Q''\in\QMS$. Send $(Q,(i,j))$ to player A and the unordered pair $(Q',(i,j))$ and $(Q'',(k,\ell))$ to player B. Accept if and only if the players' answers associated with $(Q,Q')$ are valid answers in the Magic Square test. 
   \end{enumerate}    
\end{mdframed}
\end{center}
\begin{figure}[H]
\caption{$k$-qubit EPR test~\cite{reichardt}}
\label{fig:epr_test}
\end{figure}

\paragraph{Honest EPR strategy} A $k$-qubit EPR strategy $\mS$ is a two-player strategy defined as follows. In the strategy the players share a state $\ket{\psi} = \ket{\Phi}_{\reg{A}'\reg{B}'} \otimes \ket{\psi}_{\reg{ABR}}$, where player A has registers $\reg{A}$ and $\reg{A}'$, player B has registers $\reg{B}$ and $\reg{B}'$, and register $\reg{R}$ is an arbitrary purigying register. The state $\ket{\Phi}_{\reg{A}'\reg{B}'}$ is a $k$-qubit maximally entangled state. Furthermore, each player has a set of $2k$ observables $W_i$, for $W\in \{X,Z\}$ and $i\in\{1,\ldots,k\}$, acting on its respective primed register, such that the observables satisfying the $k$-qubit Pauli (anti)-commutation relations, and the player determines its answer to any question involving $W_i$ by measuring $W_i$. 

The following follows from the results in~\cite{reichardt}. 

\begin{theorem}\label{thm:epr-test}
Let $k \geq 2$ be an integer. The $k$-qubit EPR test has the following guarantees. 
\begin{itemize}
\item (Completeness) Any $k$-qubit EPR strategy succeeds with probability $1$ in the test. 
\item (Soundness) Any strategy that succeeds with probability at least $1-\eps$ for some $\eps\geq 0$ is $\delta$-isometric to a $k$-qubit EPR strategy, for some $\delta = \poly[k;\eps]$. 
\end{itemize}
\end{theorem}

\subsection{Completeness of the Pauli check}

\begin{lemma}\label{lem:paulicheck-compleness}
Let $\mS$ be an honest Pauli Check strategy. Then the strategy is accepted with probability $1$ in the Pauli check.
\end{lemma}


\subsection{Soundness of the Pauli check}

\begin{lemma}\label{lem:paulicheck-soundness}
Any strategy $\mathcal{S}$ that passes the Pauli Check with probability at least $1 - \eps$ is $\delta$-isometric to an honest Pauli Check strategy for some $\delta = \poly[N;\eps]$.
\end{lemma}

The lemma is proved using a sequence of claims that analyze each of the tests that constitute the Pauli check. We start with the EPR test. 

\begin{claim}\label{claim:pauli-epr}
Let $\mathcal{S}$ be a strategy that passes the EPR test with probability at least $1 - \eps_1$. Then $\mathcal{S}$ is $\delta_1$-isometric to a strategy $\mathcal{S}_1$ satisfying the following properties:
\begin{enumerate}
\item On a question of the form $(W_i,W'_j)$ where $W_i$ is a Pauli question and $W'_j$ an EPR question, B determines its answer using two commuting observables $W_i$ and $W'_j$. 
\item The sub-strategy of $\mathcal{S}_1$ that consists of the shared entangled state, conditioned on $t=0$ in $\sC_p$, and the $W_i$, $i\in\{1,\ldots,2N\}$, is the honest $2N$-qubit EPR strategy, for some $\delta_1 = \poly[N;\eps_1]$, where the $2N$ shared EPR pairs are on registers $\sA'$ and $\sB'$ and the ancilla registers are $\sA$ and $\sB$. 
\item The observables $W'_j$, $j\in\{1,\ldots,2N\}$, act on $\sB$ only. 
\end{enumerate}
\end{claim}

\begin{proof}
First we observe that for any $W'_j$ that is sent as the EPR question to player $B$, player A and player B are required to successfully play a copy of the $2N$-qubit EPR test, where questions to player A do not depend on $W'_j$. Using Theorem~\ref{thm:epr-test}, this implies that the observable associated with player B's answer to the Pauli question does not depend on the EPR question, and establises item 2. in the claim. 

To show item 1. we similarly observe that the verifier's measurement of the second clock register is independent of the EPR question sent to player B. Using the consistency check performed between the verifier's measurement outcome and player B's answer to the Pauli question, we deduce that the observable associated with the Pauli question does not depend on the EPR question. 

The third item follows from the previous two, as by definition any observable for player B associated to a Pauli question approximately commutes with any observable for player B associated to an EPR question. 

\end{proof}

\begin{claim}\label{claim:pauli-propagation}
Let $\mathcal{S}_1$ be a strategy satisfying the conclusion of Claim~\ref{claim:pauli-epr}. Suppose further that $\mS_1$ passes the propagation test with probability at least $1-\eps_2$. Then $\mathcal{S}_1$ is $\delta_2$-isometric to an honest Propagation Check strategy $\mathcal{S}_2$ with the following properties:
\begin{enumerate}
\item The clock register is the register $\sC_p$. 
\item The initial state for the computation takes the form 
\begin{equation}\label{eq:pauli-prop-1}
\ket{\Phi}_{\sA'_A\sB'_B} \ket{\theta}_{\sA\sB\sZ\sR}\;,
\end{equation}
where $\ket{\Phi}_{\sA'_A\sB'_B}$ is the $2N$-qubit maximally entangled state associated with the honest EPR strategy, $\sZ$ denotes a set of registers held by the provers, and $\sR$ a purifying register.
\end{enumerate}
\end{claim}

\begin{proof}
The fact that $\mS_1$ is isometric to an honest Propagation Check strategy follows from the assumption that  $\mS_1$ passes the propagation test with probability $1-\eps_2$, and the soundness statement in Lemma~\ref{lem:prop_check} \tnote{the Lemma should be adapted a little for this to work out.}. The form of the initial state follows from the assumption that $\mS_1$ is an honest EPR strategy, so we may choose the isometry for the Propagation check so that the provers' shared state has the form~\eqref{eq:pauli-prop-1}. 
\end{proof}


\begin{claim}\label{claim:pauli-ctl}
Let $\mathcal{S}_2$ be a strategy satisfying the conclusion of Claim~\ref{claim:pauli-propagation}. Suppose further that $\mS_2$ passes the CTL check test with probability at least $1-\eps_3$. Then $\mS_2$ is $\delta_3$-close to a strategy in which the observable associated with a Pauli question $CW_i$, for $i\in\{1,\ldots,2N\}$, is the same observable as a $CTL-W_i$, for $W_i$ the obserable applied on Pauli question $W)i$, and where the control is the $t$-th qubit in $\sB'$. 
\end{claim}

\begin{proof}
Consider first the case where the verifier's outcome is $t'=0$. In this case, by assumption on $\mS_2$, the shared state is 
$$ \frac{1}{\sqrt{2}}\big(\ket{t,0}_{\reg{C}_p} \ket{\Phi}_{\reg{A}'\reg{B}'}  \ket{\theta_t} + (-1)^c  \ket{t,0}_{\reg{C}_p} \Id_{\reg{A}'} \otimes (CW_t)_{\reg{B}\reg{B}'} \ket{\Phi}_{\reg{A}'\reg{B}'} \ket{\theta_t}\big)\;.$$
In case player A reports the outcome $0$ for her EPR question $X_t$, the verifier's check implies that $CW_t$ acts as identity on $\reg{B}\reg{B}'$. In case player A reports the outcome $1$, the check implies that $CW_t$ acts as $W_t$. Thus the two observables are close, when their action on $\ket{\theta_t}$ is considered. 
\end{proof}

\begin{claim}\label{claim:pauli-relation}
Let $\mathcal{S}_3$ be a strategy satisfying the conclusion of Claim~\ref{claim:pauli-ctl}. Suppose further that $\mS_3$ passes the Pauli relation check with probability at least $1-\eps_4$. Then the observables associated with player B's Pauli questions of the form $W_i$ satisfy the conditions of Lemma~\ref{lem:gh}, up to some error $\delta_4 = \poly[N;\eps_3+\delta_3]$ and when the state is the state $\ket{\theta}$. 
\end{claim}

\begin{proof}
\tnote{todo}
\end{proof}