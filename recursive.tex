
\section{Recursive compression}

We present our recursive $\MIPstar$ compression scheme in this section. The main result we prove is the following:

\begin{theorem}
	There exists a deterministic algorithm $A$ that on input $\langle M, 1^n, R \rangle$ outputs a $(\kappa_r + 8)$-prover one-round game $G_{M,n,R}$ such that 
\begin{align*}
		\omega^*(G_{M,n,R}) &= 1  \quad & \text{if } \langle M,1^n \rangle\in \mathcal{L}_R \\
		\omega^*(G_{Mn,r,}) &\leq 1 - \exp^{R-1}_2(-cn) \quad &\text{if } \langle M,1^n \rangle\notin \mathcal{L}_R.
\end{align*}
Furthermore, $A$ runs in time $\poly(n) + \poly\log(R)$.	
\end{theorem}

Here, $\kappa_r = 2R$. \hnote{Subject to change.} The main idea is to carefully compose the $\MIPstar$ compression scheme of~\cite{ji2016compression} $R$ times, each time shrinking the completeness-soundness gap by an exponential. 



\subsection{Circuits and Turing Machines}


\paragraph{Recursive Compression VTM, $\cV_{UPC}$} This is the VTM that specifies the verifier used in our recursive compression scheme. The auxiliary input $\lambda$ is interpreted as $(G,r)$ where $G$ is a GTM and $r$ is an integer in binary. It is described in Section~\ref{sec:vtm_upc}. 

\paragraph{Universal Protocol Circuit, $UPC_{n,r}$} This is a family of $\kappa_r$-prover protocol circuits of size $\poly(n)$. The $UPC_{n,r}$ circuit takes an input $G$. It expects $G$ to be the description of a GTM for a family of protocol circuits. At a high level, $UPC_{n,r}$ is the circuit of the $\kappa_r$-prover simcom game (see Figure \ref{fig:simcom}) where the verifier circuits $(C_Q,C_A)$ are specified by $\cV_{UPC}$ that we explain in detail in Section~\ref{sec:vtm_upc}. 

Formally, let $C_Q = C_Q(n,\cV_{UPC})$ and $C_A = C_A(n,\cV_{UPC})$. Let $UPC_{n,r}$ be the circuit of the simcom game associated with $(C_Q,C_A)$. Furthermore, the register $\sV$ of $UPC_{n,r}$ has two subregisters: $\sV_{in}$ and $\sV_{work}$. The circuit receives input $G$ in $\sV_{in}$. For technical reasons, the integer $r$ is also hardcoded in binary in $\sV_{in}$ as well (this is because the circuit $C_Q$ is supposed to receive $r$ as input). Let $\tau(n,r)$ denote the circuit size of $UPC_{n,r}$. 


We use $\omega^{SC}(UPC_{n,r}(G))$ to denote the the value of the simcom game that is specified by the circuit $UPC_{n,r}$ on input $G$.

\paragraph{Gate Turing Machine, $G_{UPC}$} This is a GTM for $\{UPC_{n,r}\}$. On input $n$, $r$ and $t$ it computes the $t$-th gate of $UPC_{n,r}$. 

 \tnote{At some point we need to clarify how large $r$ is allowed to be, as a function of $n$}
\begin{lemma}
	There exists a GTM $G_{UPC}$ such that on input $(n,r,t)$, three integers written in binary and such that $1 \leq t \leq \tau(n,r)$, computes the $t$-th gate of the circuit $UPC_{n,r}$ in time $\poly\log(n) + \poly \log(r)$.
\end{lemma}

\begin{proof}
	Given $(n,r,t)$, it is straightforward to compute which of the eight steps of $UPC_{n,r}$ described above the time $t$ corresponds to. If the time $t$ corresponds to a time in the $C_Q$ or $C_A$ circuits, we invoke the GTMs from Lemma~\ref{lem:succinct_vtm_circuits}. Otherwise, all the other gates are easily seen to be efficiently computable. \hnote{Write more here...}
\end{proof}
%
%\subsection{Turing machines}
%
%\paragraph{Universal Extended Verifier GTM, $G_{UPC}$} This is a GTM for $\{UPC_{n,r}\}$. On input $n$, $r$ and $t$ it computes the $t$-th gate of $UPC_{n,r}$. 




\paragraph{Natarajan-Vidick VTM, $\cV_{NV}$} This is a VTM that is called in the final stage of the recursive composition. The auxiliary input $\lambda$ is interpeted as a non-deterministic Turing machine $M$. The VTM $V_{NV}$ computes the questions and checks the answers as required in the Natarajan-Vidick protocol (see the following Theorem) with $2$ provers to decide whether $\langle M, 1^n \rangle \in \mathcal{L}_1$.

\begin{theorem}[\cite{natarajan2017two}]\label{thm:nv}
%	$\NEXP$ is contained in $\MIPstar(2,1)$.
There exists an $\delta > 0$ such that the following holds. Let $M$ be a nondeterministic Turing machine, and let $\omega$ denote the maximum acceptance probability of the two prover protocol that is executed by the VTM $V_{NV}$ on input $(1^n,w,M)$ where $w$ is a uniformly random $n$-bit string. If $\langle M, 1^n \rangle \in \mathcal{L}_1$, then $\omega = 1$. Otherwise $\omega \leq 1 - \delta$. 
\end{theorem}


%We let $N = 2^n$. At a high level, the circuit $UPC_{n,r}$ behaves as follows:
%\begin{itemize}
%\item Generate a uniformly random string $w \in \{0,1\}^n$. 
%\item If $r=1$, execute the $k$-prover simcom game where the verifier is specified by the VTM $V_{final}$ on input $(1^n,w,M)$.
%
%\item If $r>0$, execute the $\kappa_r$-prover simcom game where the verifier is specified by the VTM $\cV_{UPC}$ on input $(1^n,w,r,G,V_{final},M)$. 

%Note that $UPC_{n,r}$ cannot compute the entire

%that corresponds to checking a history state of the $\kappa(r-1)$-prover simcom game that is specified by $CKT(G,1^N,r-1)$, with the input to that circuit being $(G,V_{final},M)$.
%for the $r$-prover extended verifier whose circuit is $CKT(G,1^N,r-1)$, with input to that circuit being $(G,V,M)$. 
%Note that $UPC_{n,r}$ cannot compute the entirety of $CKT(G,1^N,r-1)$ in $\poly(n)$ time. But it is enough for $UPC_{n,r}$ to use the procedure $G(N,r-1,t)$ for a random value of $1\leq t\leq N$.
%\end{itemize}

%We define $\omega^{SC}(UPC_{n,r})$ to denote the value of the simcom game that is described by $UPC_{n,r}$. 

\subsection{The analysis of the recursion}

\begin{restatable}{proposition}{soundness}
	Let $G_{UPC}$ be the GTM for the circuit family $\{UPC_{n,r}\}$, %let $V_{NV}$ be the Natarajan-Vidick VTM, and let $M$ be a nondeterministic Turing machine. 
	For any integer $n$ and $r$ let $\omega_{n,r}^{SC} = \omega^{SC}(UPC_{n,r}(G_{UPC}))$.
	
	Then for any integer $n$, $N=2^n$, and $r$,
	\begin{itemize}
	\item \emph{(Completeness)} If $\omega_{N,r}^{SC}=1$ then $\omega_{n,r+1}^{SC}=1$.
	\item \emph{(Soundness)} For $r \geq 1$ we have
	\[
		\omega_{n,r+1}^{SC} \,\leq\, 1 - \frac{1 - \omega_{N,r}^{SC}}{\poly(N)}.
	\]
	\end{itemize}
	\label{prop:UPC_soundness}
\end{restatable}

We defer the proof of Proposition~\ref{prop:UPC_soundness} until Section~\ref{sec:soundness}. 


\begin{corollary}
\label{cor:recursive_enl}
%Let $V_{NV}$ be the Natarajan-Vidick VTM. 
Let $\langle M,1^n \rangle$ be an instance of $\mathcal{L}_R$. There exists a constant $c > 0$ such that the simcom game specified by the circuit $UPC_{n,R}$ on input $G_{UPC}$ has value 
\begin{align*}
		&= 1  \quad & \text{if } \langle M,1^n \rangle\in \mathcal{L}_R \\
		&\leq 1 - \exp^{R-1}_2(-cn) \quad &\text{if } \langle M,1^n \rangle\notin \mathcal{L}_R.
\end{align*}
%where recall $\exp^0_a(-b) = 1/b$ by definition for $b \geq 0$.
\end{corollary}
\begin{proof}
	For all $m$, the value $\omega_{n,1}^{SC}$ when the circuit $UPC_{n,1}$ is run on input $G_{UPC}$ is $1$ if $\langle M,1^n \rangle$ is in $\mathcal{L}_1$ and is at most $1 - 1/\poly(n)$ otherwise. \hnote{Add more justification}.
	
	Applying the previous Proposition repeatedly, we obtain that if $\langle M,\exp^{R-1}_2(n) \rangle \in \mathcal{L}_1$, or equivalently if $\langle M,1^n \rangle \in \mathcal{L}_R$, then
	\[
		\omega_{n,R}^{SC} = \omega_{2^n,R-1}^{SC} = \cdots = \omega_{\exp^{R-1}_2(n),1}^{SC} = 1.
	\]
	Otherwise, if  $\langle M,1^n \rangle \notin \mathcal{L}_R$, then we have
	\begin{align*}
		\omega_{n,R}^{SC} &\leq 1 - \poly(2^{-n})(1-\omega_{2^n,R-1}^{SC}) \\
					&\leq 1 - \poly(\exp^2_2(-n))(1- \omega_{\exp^2_2(n),R-2}^{SC}) \\
					&\vdots \\
					&\leq 1 - \poly(\exp^{R-1}_2(-n))(1-\omega_{\exp^{R-1}_2(n),1}^{SC}) \\
					&\leq 1 - \exp^{R-1}_2(-cn).
	\end{align*}
\end{proof}


\begin{theorem}[Main]
	For all integer $R > 0$, 
	\[
		\NEXP^{(R)} \subseteq \MIPstar(\kappa_r+8,1)_{f_R(n)}
	\]
	 where $f_n(R) = \exp^{R-1}_2(-cn)$ for some $c > 1$.
\end{theorem}
\begin{proof}
	Fix an instance of $\langle M,1^n \rangle$ of $\mathcal{L}_R$. By Corollary~\ref{cor:recursive_enl} we have a simcom game $\mathscr{Z}_M$  of size $\poly(n,|M|)$ and involving $\kappa_r$ provers such that $\omega^{SC}(\mathscr{Z}_M) = 1$ if $\langle M,1^n \rangle \in \mathcal{L}_R$, and otherwise $\omega^{SC}(\mathscr{Z}_M) \leq 1 - f_R(n)$. 
	
	By Lemma~\ref{lem:convert_to_simcom} there exists a three-turn protocol $\Protocol_M$ such that $\omega^*(\Protocol_M) = 1$ if $\langle M,1^n \rangle \in \mathcal{L}_R$, and $\omega^*(\Protocol_M) \leq 1 - f_R(n)$ otherwise.
	
	To convert this to an $\MIPstar$ protocol, we use the compression result of~\cite{ji2016compression} as a blackbox, which gives an efficient method to transform any quantum interactive protocol $\mathscr{P}$ involving $r$ provers into a nonlocal game $G_{\mathscr{P}}$ involving $r + 8$ provers with the following property: if $\omega^*(\mathscr{P}) = 1$, then $\omega^*(G_{\mathscr{P}}) = 1$. Otherwise, 
	\[
		\omega^*(G_{\mathscr{P}}) \leq 1 - \frac{1 - \omega^*(\mathscr{P})}{\poly(n)}. 
	\]
	Since the transformation from $\langle M,1^n \rangle$ to $\Protocol_M$ is efficient and the compression reduction of~\cite{ji2016compression} is efficient, we obtain the desired Theorem statement.
	\hnote{Will probably need to make clearer...}
\end{proof}

%
%\subsection{Table of notation}
%
%For the reader's convenience, we include a table of notation.
%
%\begin{figure}[H]
%\begin{center}
%    \begin{tabular}{ | l | p{10cm} |}
%    \hline
%    \textbf{Notation} & \textbf{Explanation} \\ \hline
%    $\kappa_r$ 		& Upper bound on the number of provers that the verifier $UPC_{n,r}$, on input $(G_{UPC},V_{NV},M)$, will interact with.  \\ \hline
%    
%    $G_{UPC}(n,r,t)$ 			& The Gate Turing Machine (GTM) for the circuit family $\{ UPC_{n,r} \}$ \\ \hline
%    
%    $\cV_{UPC}(r,G,M)$			& The Verifier Turing Machine (VTM) that is used in $UPC$ \\ \hline
%    \end{tabular}
%\end{center}
%\caption{Table of notation}
%\end{figure}
%
%\begin{figure}[H]
%\begin{center}
%    \begin{tabular}{ | l | p{10cm} |}
%    \hline
%    \textbf{Register} & \textbf{Usage}\\ \hline
%    $\sC$			& Register that is prepared by provers; supposed to hold the clock register 			\\ \hline
%    $\sV_{in}$ 		& The register that stores the input string 	 \\ \hline
%    $\sV_{work}$	& The workspace register 									\\ \hline
%    $\sM_i$			& Message register that is passed to and from  the $i$'th prover	  \\ \hline
%%    $\sV_{out}$		& Stores the output							&			\\ \hline
%    \end{tabular}
%\end{center}
%\caption{Table of registers}
%\end{figure}
%


%The circuit $C_Q$ computes a $\kappa_r$-tuple of queries $(q_1,\ldots,q_{\kappa_r})$, where query $q_i$ is written on register $\sM_i$. The circuit $C_A$ acts on $\sV \sM$ and writes a single bit of output in the first qubit of register $\sV_{work}$. 


%\section{The circuit $UPC_{n,r}$ and its GTM $G_{UPC}$}
%\label{sec:upc}

%\tnote{this confused me. What is the circuit for a simcom game? Maybe this should be properly defined as the circuit in the figure? Isn't it a deterministic function of a specification of $C_Q$ and $C_A$, i.e. once these two are specified there are no more free parameters?} 


% as described in Section~\ref{sec:prelim_simcom} and depicted in Figure \ref{fig:simcom}. Associated with $UPC_{n,r}$ are two circuits $C_Q$ and $C_A$,

%Part of a simcom game are subcircuits $C_Q$, $C_A$, and $P$. In $UPC_{n,r}$, the subcircuits $C_Q$ and $C_A$ are determined by the VTM $V_{UPC}$ that we will describe fully in Section~\ref{sec:vtm_upc}. The subcircuit $P$ \tnote{what is this? What is a subcircuit? I didn't find this in the simcom section}is simply a sequence of prover reflection gates $P_1,\ldots,P_{\kappa_r}$. %where $P_i$ acts on $\sE_{1,i}' \sE_{2,i}' \sP_i \sK_{2,i}'$. 

%The registers of the circuit $UPC_{n,r}$ have the following specification.
%\begin{enumerate}
%	\item The registers $\sK_1, \sK_2$ are $2m$ qubits each and $\sE_1,\sE_1',\sE_2,\sE_2'$ are $m$ qubits wide each, where $m = O(\kappa_r \poly(n))$. 
%	\item 
%	\item The registers $\sK_1 \sK_2 \sV_{work} \sM$ are initialized in the all zeroes state. The $\sE_1 \sE_1' \sE_2 \sE_2'$ registers are initialized with maximally entangled states as described in Section~\ref{sec:prelim_simcom}. The $\sC \sP \sK_2'$ registers are initialized with an arbitrary state.

%That is, in addition to registers $\sC \sV \sM \sP$, the circuit acts on registers $\sK_1 \sK_2 \sK_2'$ and $\sE_1 \sE_2 \sE_1' \sE_2'$ as described in Section~\ref{sec:prelim_simcom}, and depicted in Figure \ref{fig:simcom}. The registers $\sK_1, \sK_2$ are $2m$ qubits wide each and $\sE_1,\sE_1',\sE_2,\sE_2'$ are $m$ qubits wide each, where $m = O(\kappa_r \log n)$.


%The circuit $UPC_{n,r}$ is specified by two subcircuits $C_Q$ and $C_A$, which have act on the $\sC \sV \sM$ registers. Since $UPC_{n,r}$ is a protocol circuit, the subcircuit $P$ that acts on registers $\sE_1' \sE_2' \sP \sK_2'$ is simply a sequence of prover reflection gates $P_1,\ldots,P_{\kappa_r}$ where $P_i$ acts on $\sE_{1,i}' \sE_{2,i}' \sP_i \sK_{2,i}'$. 

%The circuit $UPC_{n,r}$ is described in detailed pseudocode in Figure \ref{fig:upc} below.
%\tnote{I didn't get why the figure is needed. Isn't it just describing the earlier figure of a simcom game that you made? }

%\begin{center}
%\begin{mdframed}
%%    Input: The description of a GTM $G$, a VTM $V_{final}$, and a nondeterministic Turing machine $M$.
%	\begin{enumerate}
%		\item Write the integer $r$ in binary after the input triple $\lambda = (G,V,M)$ in the register $\sV_{in}$.
%		\item Apply the $C_Q(n,\cV_{UPC})$ circuit on the $\sC \sV \sM$ registers.
%    \item Apply a Bell measurement on the $\sM$ and $\sE_1$ registers, and save the measurement outcome in register $\sK_1$.
%    \item For $i \in \{1,\ldots,\kappa_r\}$, apply the prover reflection $P_i$ on the $\sE_{1,i}' \sE_{2,i}' \sP_i \sK_{2,i}'$ registers.
%    \item Apply $H^{\otimes 2m}$ (the $2m$-fold tensor product of the Hadamard gate) on $\sK_2$. 
%    \item Swap the $\sM$ and $\sE_2$ registers.
%    \item Controlled on $\sK_2$, apply the Pauli correction operation on $\sM$.
%    \item Apply the $C_A(n,V_{UPC})$ circuit on the $\sC \sV \sM$ registers.
%    \item Apply a CNOT gate to $\sO$ if $\sK_1$ is all zeroes.
%    \end{enumerate}
%\end{mdframed}
%\begin{figure}[H]
%\caption{Description of the protocol circuit \textsc{$UPC_{n,r}$}}
%\label{fig:upc}
%\end{figure}
%\end{center}
%


%verifier in an ENL game with $\kappa_r$ provers, where $\kappa_r$ is a function to be determined later (but this function will be $O(r)$). 


%The three circuits that constitute $UPC_{n,r}$ are described in Figure~\ref{fig:UPC}. Each of them is essentially an elaborate Turing machine simulator that emulates a phase of $V$. Recall the specification of the registers on which the circuits $UPC_{n,r}$ operate given at the start of Section~\ref{sec:upc}. 

%We now describe the GTM $G_{UPC}$. \tnote{At some point we need to clarify how large $r$ is allowed to be, as a function of $n$}
%
%\begin{lemma}
%	There exists a GTM $G_{UPC}$ such that on input $(n,r,t)$, three integers written in binary and such that $1 \leq t \leq \tau(n,r)$, computes the $t$-th gate of the circuit $UPC_{n,r}$ in time $\poly\log(n) + \poly \log(r)$.
%\end{lemma}
%
%\begin{proof}
%	Given $(n,r,t)$, it is straightforward to compute which of the eight steps of $UPC_{n,r}$ described above the time $t$ corresponds to. If the time $t$ corresponds to a time in the $C_Q$ or $C_A$ circuits, we invoke the GTMs from Lemma~\ref{lem:succinct_vtm_circuits}. Otherwise, all the other gates are easily seen to be efficiently computable. \hnote{Write more here...}
	
%	 Furthermore, the description of $V_{UPC}$ is hardwired into $G_{UPC}$.
%	
%	If $t$ is a gate in the measurement circuit, it is either a gate used to generate the random string $w$, or a gate used to copy inputs between different registers, or a gate of TMSIM, or a gate of CKTSIM. Which case holds can be determined easily.
%	
%	If $t$ is a gate in the questions circuit, it is either a gate used to copy inputs between different registers, or a gate of TMSIM. Which case holds can be determined easily.
%	
%	If $t$ corresponds to a query to prover $i$, then $G(n,r,t)$ will output a gate whose label is ``Query prover $i$.'' 
%	
%	If $t$ is a gate in the decision circuit, it is either a gate used to copy inputs between different registers, or a gate of TMSIM. Which case holds can be determined easily.
